\documentclass[fleqn]{article}
\usepackage[nodisplayskipstretch]{setspace}
\usepackage{amsmath, nccmath, bm}
\usepackage{amssymb}
\usepackage{enumitem}

\newcommand{\zerodisplayskip}{
	\setlength{\abovedisplayskip}{0pt}%
	\setlength{\belowdisplayskip}{0pt}%
	\setlength{\abovedisplayshortskip}{0pt}%
	\setlength{\belowdisplayshortskip}{0pt}%
	\setlength{\mathindent}{0pt}}
	
\newcommand{\norm}[1]{\left \lVert #1 \right \rVert}

\makeatletter
	\newenvironment{equationCenter}{\@fleqnfalse\begin{equation*}}{\end{equation*}}
\makeatother

\title{Homework 8}
\author{Owen Sowatzke}
\date{December 6, 2023}

\begin{document}

	\offinterlineskip
	\setlength{\lineskip}{12pt}
	\zerodisplayskip
	\maketitle
	
	\begin{enumerate}[nolistsep]
		\item Make $\mathcal{P}_2(\mathbb{R})$ into an inner product space by defining
		
		\begin{equationCenter}
			\langle p, q\rangle = \int_{0}^{1}{p(x)q(x)dx}
		\end{equationCenter}
		
		Define $T \in \mathcal{L}(\mathcal{P}_2(\mathbb{R}))$ by $T(a_0 + a_1x + a_2x^2) = a_1x$
		
		\begin{enumerate}[nolistsep]
			\item Show that T is not self-adjoint.
			
			Let $p(x) = a_0 + a_1x + a_2x^2$ and $q(x) = b_0 + b_1x + b_2x^2$
			
			For $T$ to be self-adjoint, we need $\langle Tp, q \rangle = \langle p, Tq \rangle$
			
			\begin{equation*}
				\langle Tp, q \rangle = \int_{0}^{1}{a_1x(b_0 + b_1x + b_2x^2)dx}
			\end{equation*}
			
			\begin{equation*}
				= a_1\int_{0}^{1}{(b_0x + b_1x^2 + b_2x^3)dx}
			\end{equation*}
			
			\begin{equation*}
				= a_1\left.\left(\frac{b_0x^2}{2} + \frac{b_1x^3}{3} + \frac{b_2x^4}{4}\right)\right\vert_{0}^{1}
			\end{equation*}
			
			\begin{equation*}
				= a_1\left(\frac{b_0}{2} + \frac{b_1}{3} + \frac{b_2}{4}\right)
			\end{equation*}
			
			\begin{equation*}
				\langle p, Tq \rangle = \int_{0}^{1}{(a_0 + a_1x + a_2x^2)b_1xdx}
			\end{equation*}
			
			\begin{equation*}
				= b_1\int_{0}^{1}{(a_0x + a_1x^2 + a_2x^3)dx}
			\end{equation*}
			
			\begin{equation*}
				= b_1\left.\left(\frac{a_0x^2}{2} + \frac{a_1x^3}{3} + \frac{a_2x^4}{4}\right)\right\vert_{0}^{1}
			\end{equation*}
			
			\begin{equation*}
				= b_1\left(\frac{a_0}{2} + \frac{a_1}{3} + \frac{a_2}{4}\right)
			\end{equation*}
			
			Consider $p(x) = 1$ and $q(x) = x$
			
			\begin{equation*}
				\langle Tp, q \rangle = 0
			\end{equation*}
			
			\begin{equation*}
				\langle p, Tq \rangle = \frac{1}{2}
			\end{equation*}
			
			$\therefore \langle Tp, q \rangle \neq \langle p, Tq \rangle$
			
			$\Rightarrow T$ is not self-adjoint.
			
			\item The matrix of $T$ with respect to the basis $(1,x,x^2)$ is
			
			\begin{equationCenter}
				\begin{pmatrix}
					0 & 0 & 0\\
					0 & 1 & 0\\
					0 & 0 & 0
				\end{pmatrix}.
			\end{equationCenter}
			
			This matrix equals its conjugate transpose, even though $T$ is not self-adjoint. Explain why this is not a contradiction.
			
			According to 7.10, $T^{\dag}$ is the conjugate transpose of the matrix representation of $T$ only when the matrix representation is given with respect to an orthonormal basis. The basis $(1,x,x^2)$ is not orthonormal as illustrated below:
			
			\begin{equation*}
				\langle 1, x \rangle = \int_{0}^{1}{xdx} = \left.\frac{x^2}{2}\right\vert_{0}^{1} = \frac{1}{2} \neq 0
			\end{equation*}
		\end{enumerate}
	\end{enumerate}
\end{document}


\documentclass[fleqn]{article}
\usepackage[nodisplayskipstretch]{setspace}
\usepackage{amsmath, nccmath, bm}
\usepackage{amssymb}
\usepackage{enumitem}

\newcommand{\zerodisplayskip}{
	\setlength{\abovedisplayskip}{0pt}%
	\setlength{\belowdisplayskip}{0pt}%
	\setlength{\abovedisplayshortskip}{0pt}%
	\setlength{\belowdisplayshortskip}{0pt}%
	\setlength{\mathindent}{0pt}}
	
\newcommand{\norm}[1]{\left \lVert #1 \right \rVert}

\makeatletter
	\newenvironment{equationCenter}{\@fleqnfalse\begin{equation*}}{\end{equation*}}
\makeatother

\title{Homework 8}
\author{Owen Sowatzke}
\date{December 6, 2023}

\begin{document}

	\offinterlineskip
	\setlength{\lineskip}{12pt}
	\zerodisplayskip
	\maketitle
	
	\begin{enumerate}[nolistsep]
		\item Make $\mathcal{P}_2(\mathbb{R})$ into an inner product space by defining
		
		\begin{equationCenter}
			\langle p, q\rangle = \int_{0}^{1}{p(x)q(x)dx}
		\end{equationCenter}
		
		Define $T \in \mathcal{L}(\mathcal{P}_2(\mathbb{R}))$ by $T(a_0 + a_1x + a_2x^2) = a_1x$
		
		\begin{enumerate}[nolistsep]
			\item Show that T is not self-adjoint.
			
			Let $p(x) = a_0 + a_1x + a_2x^2$ and $q(x) = b_0 + b_1x + b_2x^2$
			
			For $T$ to be self-adjoint, we need $\langle Tp, q \rangle = \langle p, Tq \rangle$
			
			\begin{equation*}
				\langle Tp, q \rangle = \int_{0}^{1}{a_1x(b_0 + b_1x + b_2x^2)dx}
			\end{equation*}
			
			\begin{equation*}
				= a_1\int_{0}^{1}{(b_0x + b_1x^2 + b_2x^3)dx}
			\end{equation*}
			
			\begin{equation*}
				= a_1\left.\left(\frac{b_0x^2}{2} + \frac{b_1x^3}{3} + \frac{b_2x^4}{4}\right)\right\vert_{0}^{1}
			\end{equation*}
			
			\begin{equation*}
				= a_1\left(\frac{b_0}{2} + \frac{b_1}{3} + \frac{b_2}{4}\right)
			\end{equation*}
			
			\begin{equation*}
				\langle p, Tq \rangle = \int_{0}^{1}{(a_0 + a_1x + a_2x^2)b_1xdx}
			\end{equation*}
			
			\begin{equation*}
				= b_1\int_{0}^{1}{(a_0x + a_1x^2 + a_2x^3)dx}
			\end{equation*}
			
			\begin{equation*}
				= b_1\left.\left(\frac{a_0x^2}{2} + \frac{a_1x^3}{3} + \frac{a_2x^4}{4}\right)\right\vert_{0}^{1}
			\end{equation*}
			
			\begin{equation*}
				= b_1\left(\frac{a_0}{2} + \frac{a_1}{3} + \frac{a_2}{4}\right)
			\end{equation*}
			
			Consider $p(x) = 1$ and $q(x) = x$
			
			\begin{equation*}
				\langle Tp, q \rangle = 0
			\end{equation*}
			
			\begin{equation*}
				\langle p, Tq \rangle = \frac{1}{2}
			\end{equation*}
			
			$\therefore \langle Tp, q \rangle \neq \langle p, Tq \rangle$
			
			$\Rightarrow T$ is not self-adjoint.
			
			\item The matrix of $T$ with respect to the basis $(1,x,x^2)$ is
			
			\begin{equationCenter}
				\begin{pmatrix}
					0 & 0 & 0\\
					0 & 1 & 0\\
					0 & 0 & 0
				\end{pmatrix}.
			\end{equationCenter}
			
			This matrix equals its conjugate transpose, even though $T$ is not self-adjoint. Explain why this is not a contradiction.
			
			According to 7.10, $T^{\dag}$ is the conjugate transpose of the matrix representation of $T$ only when the matrix representation is given with respect to an orthonormal basis. The basis $(1,x,x^2)$ is not orthonormal as illustrated below:
			
			\begin{equation*}
				\langle 1, x \rangle = \int_{0}^{1}{xdx} = \left.\frac{x^2}{2}\right\vert_{0}^{1} = \frac{1}{2} \neq 0
			\end{equation*}
		\end{enumerate}
		
		\item Suppose $\text{dim}\ V \geq 2$. Show that the set of normal operators on $V$ is not a subspace of $\mathcal{L}(V)$.
		For a normal operator, $TT^{\dag} = T^{\dag}T$
		
		For the set of normal operators to be a subspace, it must
		
		\begin{itemize}
			\item Contain the zero-vector
			\item Be closed under addition
			\item Be closed under scalar multiplication	
		\end{itemize}
		
		Provide a counter-example which shows that the set is not closed under addition.
		
		Start by considering operators on $T \in \mathcal{L}(\mathbb{R}^2)$.
		
		\begin{equation*}
			\mathcal{M}(T) = \begin{bmatrix}
				a_{11} & a_{12} \\
				a_{21} & a_{22}
			\end{bmatrix}
		\end{equation*}
		
		Consider $a_{11}$, $a_{12}$, $a_{21}$, and $a_{22}$ that make $T$ normal.
		
		\begin{equation*}
			\begin{bmatrix}
				a_{11} & a_{12} \\
				a_{21} & a_{22}
			\end{bmatrix}\begin{bmatrix}
				a_{11} & a_{21} \\
				a_{12} & a_{22}
			\end{bmatrix} = \begin{bmatrix}
				a_{11} & a_{21} \\
				a_{12} & a_{22}
			\end{bmatrix}\begin{bmatrix}
				a_{11} & a_{12} \\
				a_{21} & a_{22}
			\end{bmatrix}
		\end{equation*} 
		
		\begin{equation*}
			\begin{bmatrix}
				a_{11}^2 + a_{12}^2 & a_{11}a_{21} + a_{12}a_{22} \\
				a_{11}a_{21} + a_{12}a_{22} & a_{21}^2 + a_{22}^2
			\end{bmatrix} = \begin{bmatrix}
				a_{11}^2 + a_{21}^2 & a_{11}a_{12} + a_{21}a_{22} \\
				a_{11}a_{12} + a_{21}a_{22} & a_{12}^2 + a_{22}^2
			\end{bmatrix}
		\end{equation*}
		
		$T$ is normal when each element of the above matrices are equal.
		
		Set $a_{11} = a_{22} = 0$. Then, $T$ is normal when $a_{12}^2 = a_{21}^2$.
		
		Define an operator $S \in \mathcal{L}(V)$ by setting $a_{12} = -1$, $a_{21} = 1$, and all other elements of the matrix representation to $0$.
		
		\begin{equation*}
			S(a_1e_1 + \cdots + a_ne_n) = a_2e_1 - a_1e_2
		\end{equation*}
		
		Verify that the operator $S$ will be normal.
		
		\begin{equation*}
			S^{\dag}(a_1e_1 + \cdots + a_ne_n) = -a_2e_1 + a_1e_2
		\end{equation*}
		
		\begin{equation*}
			SS^{\dag}(a_1e_1 + \cdots + a_ne_n) = a_1e_1 + a_2e_2
		\end{equation*}
		
		\begin{equation*}
			S^{\dag}S(a_1e_1 + \cdots + a_ne_n) = a_1e_1 + a_2e_2
		\end{equation*}
		
		Because $SS^{\dag} = S^{\dag}S$, $S$ is normal.
		
		Now define a new operator $T \in \mathcal{L}(V)$ by setting $a_{12} = 1$, $a_{21} = 1$, and all other elements of the matrix representation to $0$.
		
		\begin{equation*}
			T(a_1e_1 + \cdots + a_ne_n) = a_2e_1 + a_1e_2
		\end{equation*}
		
		The operator $T$ is self-adjoint. As such, it is also normal.
		
		Now check whether $S + T$ is normal.
		
		\begin{equation*}
			(S + T)(a_1e_1 + \cdots + a_ne_n) = 2a_2e_1
		\end{equation*}
		
		\begin{equation*}
			(S + T)^{\dag}(a_1e_1 + \cdots + a_ne_n) = 2a_1e_2
		\end{equation*}
			
		\begin{equation*}
			(S + T)(S + T)^{\dag}(a_1e_1 + \cdots + a_ne_n) = 4a_1e_1
		\end{equation*}
		
		\begin{equation*}
			(S + T)^{\dag}(S + T)(a_1e_1 + \cdots + a_ne_n) = 4a_2e_2
		\end{equation*}
		
		\begin{equation*}
			(S + T)(S + T)^{\dag} \neq (S + T)^{\dag}(S + T)
		\end{equation*}
		
		$\Rightarrow S + T$ is not normal.
		
		$\therefore$, the set of normal operators on $V$ is not closed under addition, and as a result, they do not form a subspace.
		
		\item Suppose $T \in \mathcal{L}(V)$ is normal. Prove that
		
			\begin{equationCenter}
				\text{range}\ T = \text{range}\ T^{\dag}
			\end{equationCenter}
			
			Let $v \in \text{null}\ T \Rightarrow \norm{Tv} = \norm{T^{\dag}v} = 0 \Rightarrow v \in \text{null}\ T^{\dag}$
			
			Because $(T^{\dag})^{\dag} = T$, if $v \in \text{null}\ T^{\dag} \Rightarrow v \in \text{null}\ T$
			
			$\therefore \text{null}\ T = \text{null}\ T^{\dag}$
			
			$\text{range}\ T = (\text{null}\ T^{\dag})^{\perp} = (\text{null}\ T)^{\perp} = \text{range}\ T^{\dag}$
			
		\item Suppose $T \in \mathcal{L}(\mathbb{C}^2)$ is defined by $T(x,y) = (-4y, x)$. Find the singular values of $T$.
		
		\begin{equation*}
			\mathcal{M}(T) = \begin{bmatrix}
				0 & -4\\
				1 & 0
			\end{bmatrix}
		\end{equation*}
		
		\begin{equation*}
			\mathcal{M}(T^{\dag}) = \mathcal{M}(T^{\dag})^{*T} = \begin{bmatrix}
				0 & 1\\
				-4 & 0
			\end{bmatrix}
		\end{equation*}
		
		\begin{equation*}
			\mathcal{M}(TT^{\dag}) = \mathcal{M}(T^{\dag})\mathcal{M}(T) = \begin{bmatrix}
				0 & 1\\
				-4 & 0
			\end{bmatrix}\begin{bmatrix}
				0 & -4\\
				1 & 0
			\end{bmatrix} = \begin{bmatrix}
				1 & 0\\
				0 & 16
			\end{bmatrix}
		\end{equation*}
		
		Because the matrix representation of $TT^{\dag}$ is diagonal, the eigenvalues are on the main diagonal.
		
		$\lambda_1 = 1$, $\lambda_2 = 16$
		
		The singular values are the positive square roots of the eigenvalues, given from smallest to largest.
		
		$\mathbf{\therefore \sigma_1 = 4,\ \sigma_2 = 1}$
	\end{enumerate}
\end{document}


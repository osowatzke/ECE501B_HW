\documentclass[fleqn]{article}
\usepackage[nodisplayskipstretch]{setspace}
\usepackage{amsmath, nccmath}
\usepackage{amssymb}
\usepackage{enumitem}
\usepackage{etoolbox}

\newcommand{\zerodisplayskip}{
	\setlength{\abovedisplayskip}{0pt}%
	\setlength{\belowdisplayskip}{0pt}%
	\setlength{\abovedisplayshortskip}{0pt}%
	\setlength{\belowdisplayshortskip}{0pt}%
	\setlength{\mathindent}{0pt}}

\makeatletter
	\newenvironment{equationCenter}{\@fleqnfalse\begin{equation*}}{\end{equation*}}
\makeatother

\title{Homework 5}
\author{Owen Sowatzke}
\date{October 23, 2023}
\setlength{\jot}{0pt}

\begin{document}
	\offinterlineskip
	\setlength{\lineskip}{12pt}
	\zerodisplayskip
	\maketitle
	
	\begin{enumerate}[nolistsep]
		\item Define $T \in \mathcal{L}(\mathbb{F}^3)$ by
		
		\begin{equationCenter}
			T(z_1, z_2, z_3) = (2z_2, 0, 5z_3)
		\end{equationCenter}
		
		Find all eigenvalues and eigenvectors of $T$.
		
		\begin{equation*}
			\mathcal{M}(T) = \begin{bmatrix}
				0 & 2 & 0 \\
				0 & 0 & 0 \\
				0 & 0 & 5
			\end{bmatrix}
		\end{equation*}
		
		$\mathcal{M}(T)$ is an upper triangular matrix, so the eigenvalues of $T$ are $\lambda = 0, 0, 5$.
		
		Let $\lambda = 0$:
		
		The corresponding eigenvectors are in the null space of $T - {\lambda}I$. With $\lambda = 0$, this is equivalent to all the vectors in the null space of $T$.
		
		Note that all vectors in the null space of $T$ have $z_2 = z_3 = 0$. However, there are no constraints on $z_1$.
		
		$\therefore$ the eigenvectors for $\lambda = 0$ are all non-zero vectors in the following set:
		
		$\{(z_1, 0, 0) : z_1 \in \mathbb{F}\}$
		
		Let $\lambda = 5$:
		
		The corresponding eigenvectors are in the null space of $T - {\lambda}I$
		
		\begin{equation*}
			\mathcal{M}(T - {\lambda}I) = \begin{bmatrix}
				-5 &  2 & 0 \\
				 0 & -5 & 0 \\
				 0 &  0 & 0 \\
			\end{bmatrix}
		\end{equation*}
		
		\begin{equation*}
			\left\{
			\begin{aligned}
				-5z_1 + 2z_2 &= 0 \\
				-5z_2 &= 0
			\end{aligned} \right.
		\end{equation*}
		
		$\Rightarrow z_1 = z_2 = 0$
		
		Note that there are no constraints on $z_3$.
		
		$\therefore$ the eigenvectors for $\lambda = 0$ are all non-zero vectors in the following set:
		
		$\{(0, 0, z_3) : z_3 \in \mathbb{F}\}$
		
		\item Suppose $n$ is a positive integer and $T \in \mathcal{L}(\mathbb{F}^n)$ is defined by
		
			\begin{equationCenter}
				T(x_1,...,x_n) = (x_1 + \cdots + x_n,..., x_1 + \cdots + x_n);
			\end{equationCenter}
			
			in other words, $T$ is the operator whose matrix (with respect to the standard basis) consists of all 1's. Find all eigenvalues and eigenvectors of $T$.
			
			$T(x_1,...,x_n) = \lambda(x_1,...,x_n)$
			
			\begin{equation*}
			\left\{
			\begin{aligned}
				x_1 + \cdots &+ x_n &= {\lambda}x_1 \\
			    &\vdots \\
				x_1 + \cdots &+ x_n &= {\lambda}x_n
			\end{aligned} \right.
		\end{equation*}
		
			$\Rightarrow {\lambda}x_1 = \cdots = {\lambda}x_n$
			
			$\Rightarrow x_1 = \cdots = x_n$ or $\lambda = 0$.
			
			$T(x_1,...,x_n) = (x_1 + \cdots + x_n, ..., x_1 + \cdots + x_n)$
			
			$\Rightarrow T(x_1,...,x_n) = (x_1 + \cdots + x_1, ..., x_n + \cdots + x_n)$
			
			$\Rightarrow T(x_1,...,x_n) = (nx_1, ..., nx_n)$
			
			$\Rightarrow T(x_1,...,x_n) = n(x_1, ..., x_n)$
			
			$\therefore \lambda = n$ and the corresponding eigenvectors are all non-zero vectors in the following set:
			
			$\{(x_1,...,x_n) : x_1 = \cdots = x_n\}$ 
			
			\pagebreak
			Consider $\lambda = 0$:
			
			The corresponding eigenvectors are in the null space of $T - {\lambda}I$. For $\lambda = 0$, this is equivalent to the eigenvectors being in the null space of $T$.
			
			\begin{equation*}
				\mathcal{M}(T) = \begin{bmatrix}
					1 & 1 & \cdots & 1\\
					1 & 1 & \cdots & 1\\
					\vdots & \vdots & \ddots & \vdots\\
					1 & 1 & \cdots & 1\\    
				\end{bmatrix}
			\end{equation*}
			
			Using EROs to reduce $\mathcal{M}(T)$, we get:
			
			\begin{equation*}
				\begin{bmatrix}
					1 & 1 & \cdots & 1\\
					0 & 0 & \cdots & 0\\
					\vdots & \vdots & \ddots & \vdots\\
					0 & 0 & \cdots & 0\\    
				\end{bmatrix}
			\end{equation*}
			
			$\Rightarrow x_1 + \cdots + x_n = 0$
			
			$\therefore$ the eigenvectors for $\lambda = 0$ are all non-zero vectors in the following set:
		
		$\{(x_1, ..., x_n) : x_1 + \cdots + x_n = 0\}$
			
			\item Find all eigenvalues and eigenvectors of the backward shift operator \newline $T \in \mathcal{L}(F^{\infty})$ defined by
			
				\begin{equationCenter}
					T(z_1, z_2, z_3,...) = (z_2, z_3,...)
				\end{equationCenter}
				
				$z_2 = {\lambda}z_1$
				
				$z_3 = {\lambda}z_2$
				
				$\vdots$
				
				For any eigenvector $\lambda \in \mathbb{F}$, we can find an eigenvector $v \in \mathbb{F}^\infty$
				 
				 $\therefore \lambda \in \mathbb{F}$
				 
				 For each eigenvalue $\lambda$, all non-zero vectors in the following set are eigenvectors:
				 
				 $\{(x_1,{\lambda}x_1,{\lambda}^2x_1, ...) : x_1 \in \mathbb{F}\}$
				 
			\item Suppose $V$ is finite dimensional and $S,T \in \mathcal{L}(V)$. Prove that $ST$ and $TS$ have the same eigenvalues.
			
			\pagebreak
			All eigenvalues of $ST$ satisfy the following expression:
			
			$(ST)v = {\lambda}v$
			
			$\Rightarrow T((ST)v) = T({\lambda}v)$
			
			$\Rightarrow TS(Tv) = \lambda(Tv)$
			
			If $Tv \neq 0$, then $Tv$ is an eigenvector of $TS$ and $\lambda$ is the corresponding eigenvalue. For this case, $ST$ and $TS$ have the same eigenvalues.
			
			Consider $Tv = 0$, then $(ST)v = S(Tv) = 0$. If $v$ is an eigenvector of $ST$, it must be nonzero. $\therefore$ the only way that $(ST)v = {\lambda}v = 0$ is for $\lambda = 0$. In other words, $Tv = 0 \Rightarrow \lambda = 0$ is an eigenvalue of $ST$.
			
			For $\lambda = 0$ to also be a eigenvalue of $TS$, there must be a nonzero eigenvector $v$ such that $(TS)v = 0$ (i.e. $TS$ must not be injective).
			
			If we can show that $TS$ is not invertible, we can conclude that $TS$ is not injective and complete the proof.
			
			Given $Tv = 0$ with $v \neq 0$, we also know that $T$ is not injective. Since $T \in \mathcal{L}(V)$, $T$ is also not invertible. 
			
			Using a contrapositive, we can prove that if $T$ is not invertible, then $TS$ is not invertible.
			
			Statement A: $T$ is not invertible.
			
			Statement B: $TS$ is not invertible.
			
			Need to show that ${\sim}B \Rightarrow {\sim}A$
			
			Assume ${\sim}B$ ($TS$ is invertible).
			
			Then, $\exists$ $R \in \mathcal{L}(V)$ such that $R(TS) = I$ and $(TS)R = I$.
			
			Consider $v \in V$.
			
			$(TS)Rv = Iv$
			
			$T(SRv) = v$
			
			$\therefore v \in \mathcal{R}(T) \Rightarrow T$ is surjective.
			
			For $L \in \mathcal{L}(V)$, this implies that $T$ is invertible.
			
			Since we have proved ${\sim}B \Rightarrow {\sim}A$, we know that $A \Rightarrow B$ is also true.
			
			In other words, when $T$ is not invertible, $TS$ is not invertible.
			
			$\Rightarrow TS$ is not injective (i.e. $\exists\ v \neq 0$ such that $(TS)v = 0$).
			
			$\therefore$ when $\lambda = 0$ is an eigenvalue of $ST$, $\lambda = 0$ is also an eigenvalue of $TS$.
			
			Regardless of whether $Tv = 0$, all eigenvalues of $ST$ are also eigenvalues of $TS$. 
			
			The roles of $ST$ and $TS$ can be reversed to show that all eigenvalues of $TS$ are also eigenvalues of $ST$. $\therefore$ we can conclude that $ST$ and $TS$ have the same eigenvalues.
			
			\item Suppose $V$ is finite-dimensional, $T \in \mathcal{L}(V)$ has $\text{dim}\ V$ distinict eigenvalues, and $S \in \mathcal{L}(V)$ has the same eigenvectors as $T$ (not necessarily with the same eigenvalues). Prove that $ST = TS$.
			
			Because $T$ has $\text{dim}\ V$ distinct eigenvalues, it has $\text{dim}\ V$ linearly independent eigenvectors. In other words, its eigenvectors form a basis for $V$. If we let the eigenvectors of $T$ be $v_1,...,v_n$, then any $v \in V$ can be written as follows:
			
			$v = a_1v_1 + \cdots + a_nv_n$
			
			Let $\lambda_{S_1},...,\lambda_{S_n}$ be the eigenvalues of $S$, and let $\lambda_{T_1},...,\lambda_{T_n}$ be the eigenvalues of $T$. Then, 
			
			$Tv = T(av_1 + \cdots + av_n) = a_1Tv_1 + \cdots + a_1Tv_1 = a_1\lambda_{T_1}v_1 + \cdots + a_n\lambda_{T_n}v_n$
			
			$Sv = S(av_1 + \cdots + av_n) = a_1Sv_1 + \cdots + a_1Sv_1 = a_1\lambda_{S_1}v_1 + \cdots + a_n\lambda_{S_n}v_n$
			
			Now solve for $(ST)v$ and $(TS)v$:
			
			$(ST)v = S(Tv) = S(a_1\lambda_{T_1}v_1 + \cdots + a_n\lambda_{T_n}v_n)$
			
			$ = a_1\lambda_{T_1}Sv_1 + \cdots + a_n\lambda_{T_n}Sv_n = a_1\lambda_{T_1}\lambda_{S_1}v_1 + \cdots + a_n\lambda_{T_n}\lambda_{S_n}v_n$
			
			\pagebreak
			$(TS)v = T(Sv) = T(a_1\lambda_{S_1}v_1 + \cdots + a_n\lambda_{S_n}v_n)$
			
			$ = a_1\lambda_{S_1}Tv_1 + \cdots + a_n\lambda_{S_n}Tv_n = a_1\lambda_{S_1}\lambda_{T_1}v_1 + \cdots + a_n\lambda_{S_n}\lambda_{T_n}v_n$
			
			For any $v \in V$, $(ST)v = (TS)v$.
			
			$\therefore$ we can conclude that $ST = TS$.			
		\end{enumerate}	
\end{document}
\documentclass[fleqn]{article}
\usepackage[nodisplayskipstretch]{setspace}
\usepackage{amsmath, nccmath}
\usepackage{amssymb}
\usepackage{enumitem}
\usepackage{etoolbox}

\newcommand{\zerodisplayskip}{
	\setlength{\abovedisplayskip}{0pt}%
	\setlength{\belowdisplayskip}{0pt}%
	\setlength{\abovedisplayshortskip}{0pt}%
	\setlength{\belowdisplayshortskip}{0pt}%
	\setlength{\mathindent}{0pt}}

\makeatletter
	\newenvironment{equationCenter}{\@fleqnfalse\begin{equation*}}{\end{equation*}}
\makeatother

\title{Homework 5}
\author{Owen Sowatzke}
\date{October 23, 2023}

\begin{document}
	\offinterlineskip
	\setlength{\lineskip}{12pt}
	\zerodisplayskip
	\maketitle
	
	\begin{enumerate}[nolistsep]
		\item Define $T \in \mathcal{L}(\mathbb{F}^3)$ by
		
		\begin{equationCenter}
			T(z_1, z_2, z_3) = (2z_2, 0, 5z_3)
		\end{equationCenter}
		
		Find all eigenvalues and eigenvectors of $T$.
		
		\begin{equation*}
			A = \mathcal{M}(T) = \begin{bmatrix}
				0 & 2 & 0 \\
				0 & 0 & 0 \\
				0 & 0 & 5
			\end{bmatrix}
		\end{equation*}
		
		$\mathcal{M}(T)$ is an upper triangular matrix, so the eigenvalues of $T$ are $\lambda = 0, 0, 5$.
		
		Let $\lambda = 0$:
		
		The eigenvectors of $T$ are in the null space of $A - {\lambda}I$.
		
		Note that all vectors in the null space of $A - {\lambda}I$ require $z_2 = z_3 = 0$. However, there are no constraints on $z_1$.
		
		$\therefore$ there is only one linearly independent eigenvector for the repeated eigenvalue $\lambda=0$.
		
		\begin{equation*}
			v = \begin{bmatrix}
				1 \\
				0 \\
				0
			\end{bmatrix}
		\end{equation*}
		
		Let $\lambda = 5$:
		
		The eigenvectors of $T$ are in the null space of $A - {\lambda}I$.
		
		\begin{equation*}
			A - {\lambda}I = \begin{bmatrix}
				-5 &  2 & 0 \\
				 0 & -5 & 0 \\
				 0 &  0 & 0 \\
			\end{bmatrix}
		\end{equation*}
		
		Note that all vectors in the null space of $A - {\lambda}I$ require $z_1 = z_2 = 0$.
		
		However, there are no constraints on $z_1$
		
		$\therefore$ the eigenvector corresponding to $\lambda = 5$ is
		
		\begin{equation*}
			v = \begin{bmatrix}
				0 \\
				0 \\
				1 \\
			\end{bmatrix}
		\end{equation*}
		
	\end{enumerate}
\end{document}
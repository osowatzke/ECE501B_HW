\documentclass[fleqn]{article}
\usepackage[nodisplayskipstretch]{setspace}
\usepackage{amsmath, nccmath}
\usepackage{amssymb}
\usepackage{enumitem}
\usepackage{etoolbox}

\newcommand{\zerodisplayskip}{
	\setlength{\abovedisplayskip}{0pt}%
	\setlength{\belowdisplayskip}{0pt}%
	\setlength{\abovedisplayshortskip}{0pt}%
	\setlength{\belowdisplayshortskip}{0pt}%
	\setlength{\mathindent}{0pt}}

\makeatletter
	\newenvironment{equationCenter}{\@fleqnfalse\begin{equation*}}{\end{equation*}}
\makeatother

\title{Homework 5}
\author{Owen Sowatzke}
\date{October 23, 2023}
\setlength{\jot}{0pt}

\begin{document}
	\offinterlineskip
	\setlength{\lineskip}{12pt}
	\zerodisplayskip
	\maketitle
	
	\begin{enumerate}[nolistsep]
		\item Define $T \in \mathcal{L}(\mathbb{F}^3)$ by
		
		\begin{equationCenter}
			T(z_1, z_2, z_3) = (2z_2, 0, 5z_3)
		\end{equationCenter}
		
		Find all eigenvalues and eigenvectors of $T$.
		
		\begin{equation*}
			\mathcal{M}(T) = \begin{bmatrix}
				0 & 2 & 0 \\
				0 & 0 & 0 \\
				0 & 0 & 5
			\end{bmatrix}
		\end{equation*}
		
		$\mathcal{M}(T)$ is an upper triangular matrix, so the eigenvalues of $T$ are $\lambda = 0, 0, 5$.
		
		Let $\lambda = 0$:
		
		The corresponding eigenvectors are in the null space of $T - {\lambda}I$. With $\lambda = 0$, this is equivalent to all the vectors in the null space of $T$.
		
		Note that all vectors in the null space of $T$ have $z_2 = z_3 = 0$. However, there are no constraints on $z_1$.
		
		$\therefore$ the eigenvectors for $\lambda = 0$ are all non-zero vectors in the following set:
		
		$\{(z_1, 0, 0) : z_1 \in \mathbb{F}\}$
		
		Let $\lambda = 5$:
		
		The corresponding eigenvectors are in the null space of $T - {\lambda}I$
		
		\begin{equation*}
			\mathcal{M}(T - {\lambda}I) = \begin{bmatrix}
				-5 &  2 & 0 \\
				 0 & -5 & 0 \\
				 0 &  0 & 0 \\
			\end{bmatrix}
		\end{equation*}
		
		\begin{equation*}
			\left\{
			\begin{aligned}
				-5z_1 + 2z_2 &= 0 \\
				-5z_2 &= 0
			\end{aligned} \right.
		\end{equation*}
		
		$\Rightarrow z_1 = z_2 = 0$
		
		Note that there are no constraints on $z_3$.
		
		$\therefore$ the eigenvectors for $\lambda = 0$ are all non-zero vectors in the following set:
		
		$\{(0, 0, z_3) : z_3 \in \mathbb{F}\}$
		
		\item Suppose $n$ is a positive integer and $T \in \mathcal{L}(\mathbb{F}^n)$ is defined by
		
			\begin{equationCenter}
				T(x_1,...,x_n) = (x_1 + \cdots + x_n,..., x_1 + \cdots + x_n);
			\end{equationCenter}
			
			in other words, $T$ is the operator whose matrix (with respect to the standard basis) consists of all 1's. Find all eigenvalues and eigenvectors of $T$.
			
			$T(x_1,...,x_n) = \lambda(x_1,...,x_n)$
			
			\begin{equation*}
			\left\{
			\begin{aligned}
				x_1 + \cdots &+ x_n &= {\lambda}x_1 \\
			    &\vdots \\
				x_1 + \cdots &+ x_n &= {\lambda}x_n
			\end{aligned} \right.
		\end{equation*}
		
			$\Rightarrow {\lambda}x_1 = \cdots = {\lambda}x_n$
			
			$\Rightarrow x_1 = \cdots = x_n$ or $\lambda = 0$.
			
			$T(x_1,...,x_n) = (x_1 + \cdots + x_n, ..., x_1 + \cdots + x_n)$
			
			$\Rightarrow T(x_1,...,x_n) = (x_1 + \cdots + x_1, ..., x_n + \cdots + x_n)$
			
			$\Rightarrow T(x_1,...,x_n) = (nx_1, ..., nx_n)$
			
			$\Rightarrow T(x_1,...,x_n) = n(x_1, ..., x_n)$
			
			$\therefore \lambda = n$ and the corresponding eigenvectors are all non-zero vectors in the following set:
			
			$\{(x_1,...,x_n) : x_1 = \cdots = x_n\}$ 
			
			\pagebreak
			Consider $\lambda = 0$:
			
			The corresponding eigenvectors are in the null space of $T - {\lambda}I$. For $\lambda = 0$, this is equivalent to the eigenvectors being in the null space of $T$.
			
			\begin{equation*}
				\mathcal{M}(T) = \begin{bmatrix}
					1 & 1 & \cdots & 1\\
					1 & 1 & \cdots & 1\\
					\vdots & \vdots & \ddots & \vdots\\
					1 & 1 & \cdots & 1\\    
				\end{bmatrix}
			\end{equation*}
			
			Using EROs to reduce $\mathcal{M}(T)$, we get:
			
			\begin{equation*}
				\begin{bmatrix}
					1 & 1 & \cdots & 1\\
					0 & 0 & \cdots & 0\\
					\vdots & \vdots & \ddots & \vdots\\
					0 & 0 & \cdots & 0\\    
				\end{bmatrix}
			\end{equation*}
			
			$\Rightarrow x_1 + \cdots + x_n = 0$
			
			$\therefore$ the eigenvectors for $\lambda = 0$ are all non-zero vectors in the following set:
		
		$\{(x_1, ..., x_n) : x_1 + \cdots + x_n = 0\}$
			
			\item Find all eigenvalues and eigenvectors of the backward shift operator \newline $T \in \mathcal{L}(F^{\infty})$ defined by
			
				\begin{equationCenter}
					T(z_1, z_2, z_3,...) = (z_2, z_3,...)
				\end{equationCenter}
				
				$z_2 = {\lambda}z_1$
				
				$z_3 = {\lambda}z_2$
				
				$\vdots$
				
				For any eigenvector $\lambda \in \mathbb{F}$, we can find an eigenvector $v \in \mathbb{F}^\infty$
				 
				 $\therefore \lambda \in \mathbb{F}$
				 
				 For each given eigenvector, all non-zero vectors in the following set are eigenvectors:
				 
				 $\{(x_1,{\lambda}x_1,{\lambda}^2x_1, ...) : x_1 \in \mathbb{F}\}$
	\end{enumerate}
\end{document}
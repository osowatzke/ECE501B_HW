\documentclass[fleqn]{article}
\usepackage[nodisplayskipstretch]{setspace}
\usepackage{amsmath, nccmath}
\usepackage{amssymb}
\usepackage{enumitem}

\newcommand{\zerodisplayskip}{
	\setlength{\abovedisplayskip}{0pt}%
	\setlength{\belowdisplayskip}{0pt}%
	\setlength{\abovedisplayshortskip}{0pt}%
	\setlength{\belowdisplayshortskip}{0pt}%
	\setlength{\mathindent}{0pt}}
	
\newcommand{\norm}[1]{\left \lVert #1 \right \rVert}

\title{Homework 7}
\author{Owen Sowatzke}
\date{November 29, 2023}

\begin{document}
	\offinterlineskip
	\setlength{\lineskip}{12pt}
	\zerodisplayskip
	\maketitle
	
	\begin{enumerate}[nolistsep]
		\item On $\mathcal{P}_2(\mathbb{R})$, consider the inner product given by
		
		\begin{equation*}
			\langle p, q \rangle = \int_{0}^{1}{p(x)q(x)dx}
		\end{equation*}
		
		Apply the Gram-Schmidt Procedure to the basis $1, x, x^2$ to produce an orthonormal basis of $\mathcal{P}_2(\mathbb{R})$.
		
		Let $p_1(x) = 1$, $p_2(x) = x$, and $p_3(x) = x^2$.
		
		\begin{equation*}
			e_1(x) = \frac{p_1(x)}{\norm{p_1(x)}}
		\end{equation*}  
		
		\begin{equation*}
			\norm{p_1(x)}^2 = \langle p_1(x), p_1(x) \rangle = \int_{0}^{1}{dx} = \left.x\right\vert_{0}^{1} = 1
		\end{equation*}
		
		\begin{equation*}
			\Rightarrow \norm{p_1(x)} = 1			
		\end{equation*}
		
		\begin{equation*}
			\mathbf{\therefore e_1(x) = 1}
		\end{equation*}
		
		\begin{equation*}
			w_2(x) = p_2(x) - \langle p_2(x), e_1(x) \rangle e_1(x)
		\end{equation*}
		
		\begin{equation*}
			\langle p_2(x), e_1(x) \rangle = \int_{0}^{1}{x dx} = \left.\frac{x^2}{2}\right\vert_{0}^{1} = \frac{1}{2}
		\end{equation*}
		
		\begin{equation*}
			w_2(x) = x - \frac{1}{2}
		\end{equation*}
		
		\begin{equation*}
			\norm{w_2(x)}^2 = \langle w_2(x), w_2(x) \rangle = \int_{0}^{1}{\left(x - \frac{1}{2}\right)^{2}dx} = \int_{0}^{1}{\left(x^{2} - x + \frac{1}{4}\right)dx}
		\end{equation*}
		
		\begin{equation*}
			= \left.\left(\frac{x^3}{3} - \frac{x^2}{2} + \frac{x}{4}\right)\right\vert_{0}^{1} = \frac{1}{3} - \frac{1}{2} + \frac{1}{4} = \frac{4 - 6 + 3}{12} = \frac{1}{12}
		\end{equation*}
		
		\begin{equation*}
			\Rightarrow \norm{w_2(x)} = \frac{1}{2\sqrt{3}}
		\end{equation*}
			
		\begin{equation*}
			\therefore e_2(x) = \frac{w_2(x)}{\norm{w_2(x)}} = 2\sqrt{3}\left(x - \frac{1}{2}\right) = \mathbf{\sqrt{3}(2x - 1)}
		\end{equation*}
		
		\begin{equation*}
			w_3(x) = p_3(x) - \langle p_3(x), e_2(x) \rangle e_2(x) - \langle p_3(x), e_1(x) \rangle e_1(x)
		\end{equation*}
		
		\begin{equation*}
			\langle p_3(x), e_1(x) \rangle = \int_{0}^{1}{x^{2}dx} = \left.\frac{x^3}{3}\right\vert_{0}^{1} = \frac{1}{3}
		\end{equation*}
		
		\begin{equation*}
			\langle p_3(x), e_2(x) \rangle = \int_{0}^{1}{\sqrt{3}x^{2}(2x - 1)dx} = \sqrt{3}\int_{0}^{1}{(2x^{3} - x^2)dx}
		\end{equation*}
		
		\begin{equation*}
		 	= \sqrt{3}\left.\left(\frac{x^{4}}{2} - \frac{x^{3}}{3}\right)\right\vert_{0}^{1} = \sqrt{3}\left(\frac{1}{2} - \frac{1}{3}\right) = \sqrt{3}\left(\frac{3 - 2}{6}\right) = \frac{\sqrt{3}}{6}
		\end{equation*}
		
		\begin{equation*}
			\Rightarrow w_3(x) = x^2  - \left(\frac{\sqrt{3}}{6}\right)[\sqrt{3}(2x - 1)] - \frac{1}{3}(1)
		\end{equation*}
		
		\begin{equation*}
			 = x^2 - \frac{1}{2}(2x - 1) - \frac{1}{3} = x^2 - x + \frac{1}{6}
		\end{equation*}
		
		\begin{equation*}
			\norm{w_3(x)}^2 = \langle w_3(x), w_3(x) \rangle = \int_{0}^{1}{\left(x^2 - x + \frac{1}{6}\right)^{2}dx}
		\end{equation*}
		
		\begin{equation*}
			= \int_{0}^{1}{\left(x^4 - x^3 + \frac{x^2}{6} - x^3 + x^2 - \frac{x}{6} + \frac{x^2}{6} - \frac{x}{6} + \frac{1}{36}\right)dx}
		\end{equation*}
		
		\begin{equation*}
			= \int_{0}^{1}{\left(x^4 - 2x^3 + \frac{4x^2}{3} - \frac{x}{3} + \frac{1}{36}\right)dx}
		\end{equation*}
		
		\begin{equation*}
			= \left.\left(\frac{x^5}{5} - \frac{x^4}{2} + \frac{4x^3}{9} - \frac{x^2}{6} + \frac{x}{36}\right)\right\vert_{0}^{1}
		\end{equation*}
		
		\begin{equation*}
			= \frac{1}{5} - \frac{1}{2} + \frac{4}{9} - \frac{1}{6} + \frac{1}{36} = \frac{36 - 90 + 80 - 30 + 5}{180} = \frac{1}{180}
		\end{equation*}
		
		\begin{equation*}
			\Rightarrow \norm{w_3(x)} = \frac{1}{6\sqrt{5}}
		\end{equation*}
		
		\begin{equation*}
			\therefore e_3(x) = 6\sqrt{5}\left(x^2 - x + \frac{1}{6}\right) = \mathbf{ \sqrt{5}\left(6x^2 - 6x + 1\right)}
		\end{equation*}
		
		\pagebreak
		\item Find a polynomial $q \in \mathcal{P}_2(\mathbb{R})$ such that
		
		\begin{equation*}
			\int_{0}^{1}{p(x)(\cos{{\pi}x})dx} = \int_{0}^{1}{p(x)q(x)dx} 
		\end{equation*}
		
		for every $p \in \mathcal{P}_2(\mathbb{R})$.
		
		Define:
		
		\begin{equation*}
			\varphi(v) = \int_{0}^{1}{p(x)(\cos{{\pi}x})dx}
		\end{equation*}
		
		Using the basis vectors on $\mathcal{P}_2(\mathbb{R})$ found in problem 1, we can rewrite $\varphi(v)$ as follows:
		
		$\varphi(v) = \varphi(\langle v, e_1 \rangle e_1 + \langle v, e_2 \rangle e_2 + \langle v, e_3 \rangle e_3)$
		
		$ = \langle v, e_1 \rangle \varphi(e_1) + \langle v, e_2 \rangle \varphi(e_2) + \langle v, e_3 \rangle \varphi(e_3)$
		
		$ = \langle v, e_1(\varphi(e_1))^{*} + e_2(\varphi(e_2))^{*} + e_3(\varphi(e_3))^{*} \rangle$
		
		$\therefore$ the polynomial $q(x)$ is defined as follows:
		
		$q(x) = e_1(\varphi(e_1))^{*} + e_2(\varphi(e_2))^{*} + e_3(\varphi(e_3))^{*}$
		
		where:
		
		\begin{equation*}
			\varphi(e_1) = \int_{0}^{1}{(\cos{{\pi}x})dx} = \left.\frac{1}{\pi}sin({\pi}x)\right\vert_{0}^{1} = 0
		\end{equation*}
		
		\begin{equation*}
			\varphi(e_2) = \int_{0}^{1}{\sqrt{3}(2x - 1)(\cos{{\pi}x}dx})
		\end{equation*}
		
		\begin{equation*}
			 = 2\sqrt{3}\int_{0}^{1}{x(\cos{{\pi}x})dx} - \sqrt{3}\int_{0}^{1}{(\cos{{\pi}x})dx}
		\end{equation*}
		
		Let:
		
		\begin{equation*}
			u = x \Rightarrow du = dx
		\end{equation*}
		
		\begin{equation*}
			dv = \cos{{\pi}x} \Rightarrow v = \frac{1}{\pi}sin({\pi}x)
		\end{equation*}
		
		\begin{equation*}
			\int_{0}^{1}{x(\cos{{\pi}x})dx} = \left.\frac{x}{\pi}sin({\pi}x)\right\vert_{0}^{1} - \int_{0}^{1}{\frac{1}{\pi}sin({\pi}x)dx}
		\end{equation*}
		
		\begin{equation*}
			= \left.\left(\frac{x}{\pi}sin({\pi}x) + \frac{1}{{\pi}^2}cos({\pi}x)\right)\right\vert_{0}^{1} = -\frac{1}{{\pi}^2} - \frac{1}{{\pi}^2} = -\frac{2}{{\pi}^2}
		\end{equation*}
		
		\begin{equation*}
			\int_{0}^{1}{(\cos{{\pi}x})dx} = 0
		\end{equation*}
		
		\begin{equation*}
			\therefore \varphi(e_2) = -\frac{4\sqrt{3}}{{\pi}^2}
		\end{equation*}
		
		\begin{equation*}
			\varphi(e_3) = \int_{0}^{1}{\sqrt{5}(6x^2 - 6x + 1)(\cos{{\pi}x})dx}
		\end{equation*}
		
		\begin{equation*}
			= 6\sqrt{5}\int_{0}^{1}{x^2(\cos{{\pi}x})dx} - 6\sqrt{5}\int_{0}^{1}{x(\cos{{\pi}x})dx} + \sqrt{5}\int_{0}^{1}{(\cos{{\pi}x})dx}
		\end{equation*}
		
		Let:
		
		\begin{equation*}
			u = x^2 \Rightarrow du = 2xdx
		\end{equation*}
		
		\begin{equation*}
			dv = \cos{{\pi}x} \Rightarrow v = \frac{1}{\pi}\sin({\pi}x)
		\end{equation*}
		
		\begin{equation*}
			\int_{0}^{1}{x^2(\cos{{\pi}x})dx} = \left.\frac{x^2}{\pi}sin({\pi}x)\right\vert_{0}^{1} - \frac{2}{\pi}\int_{0}^{1}{x\sin({\pi}x)dx}
		\end{equation*}
		
		Let:
		
		\begin{equation*}
			u = x \Rightarrow du = dx
		\end{equation*}
		
		\begin{equation*}
			dv = \sin{({\pi}x)} \Rightarrow v = -\frac{1}{\pi}\cos{({\pi}x)}
		\end{equation*}
		
		\begin{equation*}
			\int_{0}^{1}{x\sin({\pi}x)dx} = -\left.\frac{x}{\pi}\cos{({\pi}x)}\right\vert_{0}^{1} + \int_{0}^{1}{\frac{1}{\pi}\cos{({\pi}x)}dx}
		\end{equation*}
		
		\begin{equation*}
			= \left.\left(-\frac{x}{\pi}\cos{({\pi}x)} + \frac{1}{{\pi}^2}\sin{({\pi}x)}dx}\right)\right\vert_{0}^{1}
		\end{equation*}
		
		\begin{equation*}
			\Rightarrow \int_{0}^{1}{x^2(\cos{{\pi}x})dx} = \left.\frac{x^2}{\pi}sin({\pi}x)\right\vert_{0}^{1} - \frac{2}{\pi}\int_{0}^{1}{x\sin({\pi}x)dx}
		\end{equation*}
	\end{enumerate}
\end{document}
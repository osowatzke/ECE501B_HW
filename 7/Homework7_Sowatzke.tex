\documentclass[fleqn]{article}
\usepackage[nodisplayskipstretch]{setspace}
\usepackage{amsmath, nccmath}
\usepackage{amssymb}
\usepackage{enumitem}

\newcommand{\zerodisplayskip}{
	\setlength{\abovedisplayskip}{0pt}%
	\setlength{\belowdisplayskip}{0pt}%
	\setlength{\abovedisplayshortskip}{0pt}%
	\setlength{\belowdisplayshortskip}{0pt}%
	\setlength{\mathindent}{0pt}}
	
\newcommand{\norm}[1]{\left \lVert #1 \right \rVert}

\title{Homework 7}
\author{Owen Sowatzke}
\date{November 29, 2023}

\begin{document}
	\offinterlineskip
	\setlength{\lineskip}{12pt}
	\zerodisplayskip
	\maketitle
	
	\begin{enumerate}[nolistsep]
		\item On $\mathcal{P}_2(\mathbf{R})$, consider the inner product given by
		
		\begin{equation*}
			\langle p, q \rangle = \int_{0}^{1}{p(x)q(x)dx}
		\end{equation*}
		
		Apply the Gram-Schmidt Procedure to the basis $1, x, x^2$ to produce an orthonormal basis of $\mathcal{P}_2(\mathbf{R})$.
		
		Let $p_1(x) = 1$, $p_2(x) = x$, and $p_3(x) = x^2$.
		
		\begin{equation*}
			v_1(x) = \frac{p_1(x)}{\norm{p_1(x)}}
		\end{equation*}  
		
		\begin{equation*}
			\norm{p_1(x)}^2 = \langle p_1(x), p_1(x) \rangle = \int_{0}^{1}{dx} = \left.x\right\vert_{0}^{1} = 1
		\end{equation*}
		
		\begin{equation*}
			\Rightarrow \norm{p_1(x)} = 1			
		\end{equation*}
		
		\begin{equation*}
			\mathbf{\therefore v_1(x) = 1}
		\end{equation*}
		
	\end{enumerate}
\end{document}
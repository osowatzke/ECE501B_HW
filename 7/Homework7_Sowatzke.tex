\documentclass[fleqn]{article}
\usepackage[nodisplayskipstretch]{setspace}
\usepackage{amsmath, nccmath}
\usepackage{amssymb}
\usepackage{enumitem}

\newcommand{\zerodisplayskip}{
	\setlength{\abovedisplayskip}{0pt}%
	\setlength{\belowdisplayskip}{0pt}%
	\setlength{\abovedisplayshortskip}{0pt}%
	\setlength{\belowdisplayshortskip}{0pt}%
	\setlength{\mathindent}{0pt}}
	
\newcommand{\norm}[1]{\left \lVert #1 \right \rVert}

\title{Homework 7}
\author{Owen Sowatzke}
\date{November 29, 2023}

\begin{document}
	\offinterlineskip
	\setlength{\lineskip}{12pt}
	\zerodisplayskip
	\maketitle
	
	\begin{enumerate}[nolistsep]
		\item On $\mathcal{P}_2(\mathbf{R})$, consider the inner product given by
		
		\begin{equation*}
			\langle p, q \rangle = \int_{0}^{1}{p(x)q(x)dx}
		\end{equation*}
		
		Apply the Gram-Schmidt Procedure to the basis $1, x, x^2$ to produce an orthonormal basis of $\mathcal{P}_2(\mathbf{R})$.
		
		Let $p_1(x) = 1$, $p_2(x) = x$, and $p_3(x) = x^2$.
		
		\begin{equation*}
			e_1(x) = \frac{p_1(x)}{\norm{p_1(x)}}
		\end{equation*}  
		
		\begin{equation*}
			\norm{p_1(x)}^2 = \langle p_1(x), p_1(x) \rangle = \int_{0}^{1}{dx} = \left.x\right\vert_{0}^{1} = 1
		\end{equation*}
		
		\begin{equation*}
			\Rightarrow \norm{p_1(x)} = 1			
		\end{equation*}
		
		\begin{equation*}
			\mathbf{\therefore e_1(x) = 1}
		\end{equation*}
		
		\begin{equation*}
			w_2(x) = p_2(x) - \langle p_2(x), e_1(x) \rangle e_1(x)
		\end{equation*}
		
		\begin{equation*}
			\langle p_2(x), e_1(x) \rangle = \int_{0}^{1}{x dx} = \left.\frac{x^2}{2}\right\vert_{0}^{1} = \frac{1}{2}
		\end{equation*}
		
		\begin{equation*}
			w_2(x) = x - \frac{1}{2}
		\end{equation*}
		
		\begin{equation*}
			\norm{w_2(x)}^2 = \langle w_2(x), w_2(x) \rangle = \int_{0}^{1}{\left(x - \frac{1}{2}\right)^{2}dx} = \int_{0}^{1}{\left(x^{2} - x + \frac{1}{4}\right)dx}
		\end{equation*}
		
		\begin{equation*}
			= \left.\frac{x^3}{3} - \frac{x^2}{2} + \frac{x}{4}\right\vert_{0}^{1} = \frac{1}{3} - \frac{1}{2} + \frac{1}{4} = \frac{4 - 6 + 3}{12} = \frac{1}{12}
		\end{equation*}
		
		\begin{equation*}
			\Rightarrow \norm{w_2(x)} = \frac{1}{2\sqrt{3}}
		\end{equation*}
			
		\begin{equation*}
			\therefore e_2(x) = \frac{w_2(x)}{\norm{w_2(x)}} = 2\sqrt{3}\left(x - \frac{1}{2}\right) = \mathbf{\sqrt{3}(2x - 1)}
		\end{equation*}
		
	\end{enumerate}
\end{document}
\documentclass[fleqn]{article}

\usepackage{geometry}
\usepackage{amsmath, nccmath}
\usepackage{amssymb}
\usepackage{graphicx}
\usepackage{enumitem}
\usepackage[nodisplayskipstretch]{setspace}
\usepackage{float}

\title{Homework 2}
\author{Owen Sowatzke}
\date{September 13, 2023}
	
\begin{document}

	\doublespacing
	\setlength{\abovedisplayskip}{0pt}
	\setlength{\belowdisplayskip}{0pt}
	\setlength{\abovedisplayshortskip}{0pt}
	\setlength{\belowdisplayshortskip}{0pt}
	\setlength{\mathindent}{0pt}
	\maketitle
				
	\begin{enumerate}[nolistsep]
	
		\item[2.A.15] Prove that $\mathbb{F}^\infty$ is infinite-dimensional.
		
		Consider any list of elements of $\mathbb{F}^\infty$. Each element of the list is a vector $v = (x_1, x_2, ...) \in \mathbb{F}^\infty$. Find the largest index $m$, which corresponds to a nonzero coordinate $x_m$ in at least one of the vectors.
		
		 Each element of the list is in the span of the following vectors:
		 
		 $v_1 = (1, 0, 0, ...)$
		 
		 $v_2 = (0, 1, 0, ...)$
		 
		 $\vdots$
		 
		 $v_m = (0, .., 0, 1, 0, ...)$
		 
		 where $v_i$ is the vector where the i-th element is $1$ and all other elements are $0$.
		 
		 Now choose a vector $w \in \mathbb{F}^\infty$ where the $m + 1$ element is $1$ and all other elements are $0$.
		 
		 $w \not\in \text{span}(v_1, ..., v_m)$.
		 
		 $w$ can be added to the list of spanning vectors. Now, $w \in span(v_1, ..., v_m, w)$. However, no matter how many times the list of spanning vectors is extended, there will always be a vector in $\mathbb{F}^\infty$ that is not spanned by the list. $\therefore$ no list spans $\mathbb{F}^\infty$. Because no list spans $\mathbb{F}^\infty$, $\mathbb{F}^\infty$ is infinite-dimensional.
		
		\item[2.B.3]
		
		 	\begin{enumerate}[nolistsep]
		 	
		 		\item[(a)] Let $U$ be the subspace of $\mathbb{R}^5$ defined by
		
				$U = \{(x_1, x_2, x_3, x_4, x_5) \in \mathbb{R}^5 : x_1 = 3x_2\ \text{and}\ x_3 = 7x_4\}$
		
				Find a basis of $U$.
				
				$U = \{(3x_2, x_2, 7x_4, x_4, x_5) \in \mathbb{R}^5 : x_2, x_4, x_5 \in \mathbb{R}\}$
				
				Every vector $v \in U$ can be expressed as a linear combination of vectors as follows:
				
				$v = x_2v_1 + x_4v_2 + x_5v_3$
				
				$(3x_2, x_2, 7x_4, x_4, x_5) = x_2(3, 1, 0, 0, 0) + x_4(0, 0, 7, 1, 0) + x_5(0, 0, 0, 0, 1)$
				
				$\therefore v_1, v_2, v_3$ span $U$.
									
				The only way to make $v = 0$ is to set $x_1 = x_2 = x_3 = 0$.
				
				$\therefore v_1, v_2, v_3$ are linearly independent.
				
				Because $v_1, v_2, v_3$ are linearly independent and they span $U$, they form a basis for $U$.
				
				$\therefore (3, 1, 0, 0, 0), (0, 0, 7, 1, 0), (0, 0, 0, 0, 1)$ are basis vectors for $U$.
			\end{enumerate}
	\end{enumerate}
	
\end{document}
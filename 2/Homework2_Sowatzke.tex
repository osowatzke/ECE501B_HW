\documentclass[fleqn]{article}

\usepackage{geometry}
\usepackage{amsmath, nccmath}
\usepackage{amssymb}
\usepackage{graphicx}
\usepackage{enumitem}
\usepackage[nodisplayskipstretch]{setspace}
\usepackage{float}

\title{Homework 2}
\author{Owen Sowatzke}
\date{September 13, 2023}
	
\begin{document}

	\doublespacing
	\setlength{\abovedisplayskip}{0pt}
	\setlength{\belowdisplayskip}{0pt}
	\setlength{\abovedisplayshortskip}{0pt}
	\setlength{\belowdisplayshortskip}{0pt}
	\setlength{\mathindent}{0pt}
	\maketitle
				
	\begin{enumerate}[nolistsep]
	
		\item[2.A.15] Prove that $\mathbb{F}^\infty$ is infinite-dimensional.
		
		Consider any list of elements of $\mathbb{F}^\infty$. Each element of the list is a vector $v = (x_1, x_2, ...) \in \mathbb{F}^\infty$. Find the largest index $m$, which corresponds to a nonzero coordinate $x_m$ in at least one of the vectors.
		
		 Each element of the list is in the span of the following vectors:
		 
		 $v_1 = (1, 0, 0, ...)$
		 
		 $v_2 = (0, 1, 0, ...)$
		 
		 $\vdots$
		 
		 $v_m = (0, .., 0, 1, 0, ...)$
		 
		 where $v_i$ is the vector where the i-th element is $1$ and all other elements are $0$.
		 
		 Now choose a vector $w \in \mathbb{F}^\infty$ where the $m + 1$ element is $1$ and all other elements are $0$.
		 
		 $w \not\in \text{span}(v_1, ..., v_m)$.
		 
		 $w$ can be added to the list of spanning vectors. Now, $w \in span(v_1, ..., v_m, w)$. However, no matter how many times the list of spanning vectors is extended, there will always be a vector in $\mathbb{F}^\infty$ that is not spanned by the list. $\therefore$ no list spans $\mathbb{F}^\infty$. Because no list spans $\mathbb{F}^\infty$, $\mathbb{F}^\infty$ is infinite-dimensional.
		
		\item[2.B.3]
		
		 	\begin{enumerate}[nolistsep]
		 	
		 		\item[(a)] Let $U$ be the subspace of $\mathbb{R}^5$ defined by
		
				$U = \{(x_1, x_2, x_3, x_4, x_5) \in \mathbb{R}^5 : x_1 = 3x_2\ \text{and}\ x_3 = 7x_4\}$
		
				Find a basis of $U$.
				
				$U = \{(3x_2, x_2, 7x_4, x_4, x_5) \in \mathbb{R}^5 : x_2, x_4, x_5 \in \mathbb{R}\}$
				
				Every vector $v \in U$ can be expressed as a linear combination of vectors as follows:
				
				$v = x_2v_1 + x_4v_2 + x_5v_3$
				
				$(3x_2, x_2, 7x_4, x_4, x_5) = x_2(3, 1, 0, 0, 0) + x_4(0, 0, 7, 1, 0) + x_5(0, 0, 0, 0, 1)$
				
				$\therefore v_1, v_2, v_3$ span $U$.
									
				The only way to make $v = 0$ is to set $x_1 = x_2 = x_3 = 0$.
				
				$\therefore v_1, v_2, v_3$ are linearly independent.
				
				Because $v_1, v_2, v_3$ are linearly independent and they span $U$, they form a basis for $U$.
				
				$\therefore (3, 1, 0, 0, 0), (0, 0, 7, 1, 0), (0, 0, 0, 0, 1)$ are basis vectors for $U$.
				
				\item[(b)] Extend the basis in part (a) to a basis of $\mathbb{R}^5$.
				
				First, extend the basis to a list that spans $\mathbb{R}^5$. This can be done by appending a basis of $\mathbb{R}^5$ onto the end of the basis from part (a).
				
				Then, reduce the spanning list to form a basis.
				
				Since the basis of $U$ is at the beginning of the list, none of the vectors from the initial basis will be removed. Instead, only appended vectors will be removed from the list.
				
				Consider the following basis for $\mathbb{R}^5$:
				
				$(1, 0, 0, 0, 0), (0, 1, 0, 0, 0), (0, 0, 1, 0, 0), (0, 0, 0, 1, 0), (0, 0, 0, 0, 1)$
				
				After appending the basis for $\mathbb{R}^5$ onto the end of the basis for $U$, the following list of vectors results:
				
				$(3, 1, 0, 0, 0), (0, 0, 7, 1, 0), (0, 0, 0, 0, 1), (1, 0, 0, 0, 0), (0, 1, 0, 0, 0), (0, 0, 1, 0, 0),$
				
				$(0, 0, 0, 1, 0), (0, 0, 0, 0, 1)$
				
				Denote the vectors in the list as $v_1, v_2, ..., v_8$
				
				Now the spanning list can be reduced to form a basis.
				
				Start with $B$ equal to the list $v_1, v_2, ..., v_8$.
				
				$v_1 = (3, 1, 0, 0, 0) \neq 0$ so leave $B$ unchanged
				
				$v_2 = (0, 0, 7, 1, 0) \not\in \text{span}(v_1)$ so leave $B$ unchanged
				
				$v_3 = (0, 0, 0, 0, 1) \not\in \text{span}(v_1, v_2)$ so leave $B$ unchanged.
				
				$v_4 = (1, 0, 0, 0, 0) \not\in \text{span}(v_1, v_2, v_3)$ so leave $B$ unchanged.
				
				$v_5 = (0, 1, 0, 0, 0) = v_1 - 3v_4 \Rightarrow v_5 \in \text{span}(v_1, v_2, v_3, v_4)$ so remove $v_5$ from $B$
				
				$v_6 = (0, 0, 1, 0, 0) \not\in \text{span}(v_1, v_2, v_3, v_4)$ so leave $B$ unchanged
				
				$v_7 = (0, 0, 0, 1, 0) = v_2 - 7v_6 \Rightarrow v_7 \in \text{span}(v_1, v_2, v_3, v_4, v_6)$ so remove $v_7$ from $B$
				
				$v_8 = (0, 0, 0, 0, 1) = v_3 \Rightarrow v_8 \in \text{span}(v_1, v_2, v_3, v_4, v_6)$ so remove $v_8$ from $B$
				 
				The remaining list $B$ is a basis for $\mathbb{R}^5$ and is given by:
				
				$(3, 1, 0, 0, 0), (0, 0, 7, 1, 0), (0, 0, 0, 0, 1), (1, 0, 0, 0, 0), (0, 0, 1, 0, 0)$
					
				\item[(c)] Find a subspace $W$ of $\mathbb{R}^5$ such that $\mathbb{R}^5 = U \oplus W$.
				
				Let $V = \mathbb{R}^5$
				
				From part (b), $u_1, u_2, u_3, w_1, w_2$ is a basis of $V$ where:
				
				$u_1 = (3, 1, 0, 0, 0)$
				
				$u_2 = (0, 0, 7, 1, 0)$
				
				$u_3 = (0, 0, 0, 0, 1)$
				
				$w_1 = (1, 0, 0, 0, 0)$
				
				$w_2 = (0, 0, 1, 0, 0)$
				
				For $v \in V$
				
				\begin{align*}
				v = \underbrace{a_1u_1 + a_2u_2 + a_3u_3}_{u} + \underbrace{b_1w_1 + b_2w_2}_{w}
				\end{align*}
				
				$v = u + w$ for $u \in U$ and $v \in V$
				
				$\Rightarrow v \in U + W$
				
				$\Rightarrow V = U + W$
				
				Let $v \in U \cap W$
				
				$v = a_1u_1 + a_2u_2 + a_3u_3 = b_1w_1 + b_2w_2$
				
				Because $u_1, u_2, u_3, w_1, w_2$ is a basis for $V$, it is linearly independent.
				
				$\therefore$ the only solution to
				
				$a_1u_1 + a_2u_2 + a_3u_3 - b_1w_1 - b_2w_2 = 0$
				
				is $a_1 = a_2 = a_3 = b_1 = b_2 = 0$
				
				$\Rightarrow v = 0$.
				
				$\therefore U \cap W = \{0\}$
				
				We can then conclude that $U \oplus W = V$.
				
				Now we just need to solve for $W$.
				
				$W = \text{span}(w_1, w_2)$
				
				$\Rightarrow W = \{(x_1, 0, x_3, 0, 0) \in \mathbb{R}^5 : x_1, x_3 \in \mathbb{R}\}$				   		
			\end{enumerate}
			
		\item[2.C.11] Suppose that $U$ and $W$ are subspaces of $\mathbb{R}^8$ such that $\text{dim}\ U = 3$, $\text{dim}\ W = 5$, and $U + W = \mathbb{R}^8$. Prove that $\mathbb{R}^8 = U \oplus W$.
		
		$\text{dim}(U + W) = \text{dim}\ U + \text{dim}\ W - \text{dim}(U \cap W)$
		
		$\text{dim}(\mathbb{R}^8) = \text{dim}\ U + \text{dim}\ W - \text{dim}(U \cap W)$
		
		$8 = 3 + 5 - \text{dim}(U \cap W)$
		
		$\text{dim}(U \cap W) = 0$
		
		$\therefore U \cap W = \{0\}$
		
		Because $U \cap W = \{0\}$, $U + W$ is a direct sum and $U \oplus W = \mathbb{R}^8$.
		
		\item[2.C.12] Suppose $U$ and $W$ are both five-dimensional subspaces of $\mathbb{R}^9$. Prove that $U \cap W \neq \{0\}$.
		
		$U + W$ is a subspace of $\mathbb{R}^9$.
		
		$\therefore \text{dim}(U + W) \leq \text{dim}(R^9)$
		
		$\text{dim}\ U + \text{dim}\ W - \text{dim}(U \cap W) = \text{dim}(U + W) \leq \text{dim}(R^9)$
		
		$5 + 5 - \text{dim}(U \cap W) \leq 9$
		
		$10 - \text{dim}(U \cap W) \leq 9$
		
		$-\text{dim}(U \cap W) \leq -1$
		
		$\text{dim}(U \cap W) \geq 1$
		
		$\therefore \text{dim}(U \cap W) \neq 0 \Rightarrow U \cap W \neq \{0\}$.
	\end{enumerate}
	
\end{document}
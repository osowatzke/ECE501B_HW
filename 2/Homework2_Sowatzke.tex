\documentclass[fleqn]{article}

\usepackage{geometry}
\usepackage{amsmath, nccmath}
\usepackage{amssymb}
\usepackage{graphicx}
\usepackage{enumitem}
\usepackage[nodisplayskipstretch]{setspace}
\usepackage{float}

\title{Homework 2}
\author{Owen Sowatzke}
\date{September 13, 2023}
	
\begin{document}

	\doublespacing
	\setlength{\abovedisplayskip}{0pt}
	\setlength{\belowdisplayskip}{0pt}
	\setlength{\abovedisplayshortskip}{0pt}
	\setlength{\belowdisplayshortskip}{0pt}
	\setlength{\mathindent}{0pt}
	\maketitle
				
	\begin{enumerate}[nolistsep]
	
		\item[2.A.15] Prove that $\mathbb{F}^\infty$ is infinite-dimensional.
		
		Consider any list of elements of $\mathbb{F}^\infty$. Each element of the list is a vector $v = (x_1, x_2, ...) \in \mathbb{F}^\infty$. Find the largest index $m$, which corresponds to a nonzero coordinate $x_m$ in at least one of the vectors.
		
		 Each element of the list is in the span of the following vectors:
		 
		 $v_1 = (1, 0, 0, ...)$
		 
		 $v_2 = (0, 1, 0, ...)$
		 
		 $\vdots$
		 
		 $v_m = (0, .., 0, 1, 0, ...)$
		 
		 where $v_i$ is the vector where the i-th element is $1$ and all other elements are $0$.
		 
		 Now choose a vector $v_{m+1} \in \mathbb{F}^\infty$ where the $m + 1$ element is $1$ and all other elements are $0$.
		 
		 $v_{m+1} \not\in \text{span}(v_1, ..., v_m)$.
		 
		 $v_{m+1}$ can be added to the list of spanning vectors. Now, $v_{m+1} \in span(v_1, ..., v_m, v_{m+1})$. However, no matter how many times the list of spanning vectors is extended, there will always be a vector in $\mathbb{F}^\infty$ that is not spanned by the list. $\therefore$ no list spans $\mathbb{F}^\infty$. Because no list spans $\mathbb{F}^\infty$, $\mathbb{F}^\infty$ is infinite-dimensional.
		
		\item[2.B.3]
		
		 	\begin{enumerate}[nolistsep]
		 	
		 		\item[(a)] Let $U$ be the subspace of $\mathbb{R}^5$ defined by
		
				$U = \{(x_1, x_2, x_3, x_4, x_5) \in \mathbb{R}^5 : x_1 = 3x_2\ \text{and}\ x_3 = 7x_4\}$
		
				Find a basis of $U$.
				
				$U = \{(3x_2, x_2, 7x_4, x_4, x_5) \in \mathbb{R}^5 : x_2, x_4, x_5 \in \mathbb{R}\}$
				
				Every vector $u \in U$ can be expressed as a linear combination of vectors as follows:
				
				$u = x_2u_1 + x_4u_2 + x_5u_3$
				
				$(3x_2, x_2, 7x_4, x_4, x_5) = x_2(3, 1, 0, 0, 0) + x_4(0, 0, 7, 1, 0) + x_5(0, 0, 0, 0, 1)$
				
				$\therefore u_1, u_2, u_3$ span $U$.
									
				The only way to make $u = 0$ is to set $x_1 = x_2 = x_3 = 0$.
				
				$\therefore u_1, u_2, u_3$ are linearly independent.
				
				Because $u_1, u_2, u_3$ are linearly independent and they span $U$, they form a basis for $U$.
				
				$\therefore (3, 1, 0, 0, 0), (0, 0, 7, 1, 0), (0, 0, 0, 0, 1)$ are basis vectors for $U$.
				
				\item[(b)] Extend the basis in part (a) to a basis of $\mathbb{R}^5$.
				
				Append an $\mathbb{R}^5$ basis onto the end of the existing set of basis vectors:
				
				$\underbrace{(3,1,0,0,0),(0,0,7,1,0),(0,0,0,0,1)}_{\text{basis for } U},\underbrace{(1,0,0,0,0),...,(0,0,0,0,1)}_{\text{basis for } \mathbb{R}^5}$
				
				Denote the vectors in the resulting list as $u_1, u_2, u_3, w_1, ..., w_5$.
				
				$\text{span}(u_1, u_2, u_3, w_1, ..., w_5) = \mathbb{R}^5$ so $u_1, u_2, u_3, w_1, ..., w_5$ spans $\mathbb{R}^5$
				
				Now reduce the spanning list to form a basis.
				
				Note: None of the $u$'s will be deleted during this procedure, so the resulting basis will be $u_1, u_2, u_3$ and some of the $w$'s. $\therefore$ after performing this procedure, we will have extended the basis of $U$ to a basis of $\mathbb{R}^5$.
				
				Start with $B$ equal to the list $u_1, u_2, u_3, w_1, ..., w_5$.
				
				Step 1: 
				
				$u_1 = (3, 1, 0, 0, 0) \neq 0$ so leave $B$ unchanged
				
				Step 2: 
				
				$u_2 = (0, 0, 7, 1, 0) \not\in \text{span}(u_1)$ so leave $B$ unchanged
				
				Step 3: 
				
				$u_3 = (0, 0, 0, 0, 1) \not\in \text{span}(u_1, u_2)$ so leave $B$ unchanged.
				
				Step 4: 
				
				$w_1 = (1, 0, 0, 0, 0) \not\in \text{span}(u_1, u_2, u_3)$ so leave $B$ unchanged.
				
				Step 5: 
				
				$w_2 = (0, 1, 0, 0, 0) = u_1 - 3w_1 \Rightarrow w_2 \in \text{span}(u_1, u_2, u_3, w_1)$ so remove $w_2$ from $B$
				
				Step 6: 
				
				$w_3 = (0, 0, 1, 0, 0) \not\in \text{span}(u_1, u_2, u_3, w_1)$ so leave $B$ unchanged
				
				Step 7: 
				
				$w_4 = (0, 0, 0, 1, 0) = u_2 - 7w_3 \Rightarrow w_4 \in \text{span}(u_1, u_2, u_3, w_1, w_3)$ so remove $w_4$ from $B$
				
				Step 8: 
				
				$w_5 = (0, 0, 0, 0, 1) = u_3 \Rightarrow w_5 \in \text{span}(u_1, u_2, u_3, w_1, w_3)$ so remove $w_5$ from $B$
				
				The resulting list $B$ is a basis for $\mathbb{R}^5$. i.e. 
				
				$(3, 1, 0, 0, 0), (0, 0, 7, 1, 0), (0, 0, 0, 0, 1), (1, 0, 0, 0, 0), (0, 0, 1, 0, 0)$
				
				is a basis for $\mathbb{R}^5$
					
				\item[(c)] Find a subspace $W$ of $\mathbb{R}^5$ such that $\mathbb{R}^5 = U \oplus W$.
				
				From part (b), we know that $u_1, u_2, u_3, w_1, w_3$ is a basis of $V$.
				
				$\therefore$ for $v \in \mathbb{R}^5$
				
				$v = \underbrace{a_1u_1 + a_2u_2 + a_3u_3}_{u} + \underbrace{b_1w_1 + b_3w_3}_{w}$
				
				$\Rightarrow v = u + w$ for $u \in U$ and $w \in W$
				
				$\Rightarrow v \in U + W$
				
				$\Rightarrow \mathbb{R}^5 = U + W$
				
				Let $v \in U \cap W$
				
				$v = a_1u_1 + a_2u_2 + a_3u_3 = b_1w_1 + b_3w_3$
				
				Because $u_1, u_2, u_3, w_1, w_3$ is a basis for $\mathbb{R}^5$, it is linearly independent.
				
				$\therefore$ the only solution to
				
				$a_1u_1 + a_2u_2 + a_3u_3 - b_1w_1 - b_3w_3 = 0$
				
				is $a_1 = a_2 = a_3 = b_1 = b_3 = 0$
				
				$\Rightarrow v = 0$.
				
				$\therefore U \cap W = \{0\}$
				
				We have now found an equation for $W$ such that $\mathbb{R}^5 = U \oplus W$.
				
				Specifically, $W = \text{span}(w_1, w_3)$
				
				$\Rightarrow W = \{(x_1, 0, x_3, 0, 0) \in \mathbb{R}^5 : x_1, x_3 \in \mathbb{R}\}$				   		
			\end{enumerate}
			
		\item[2.C.11] Suppose that $U$ and $W$ are subspaces of $\mathbb{R}^8$ such that $\text{dim}\ U = 3$, $\text{dim}\ W = 5$, and $U + W = \mathbb{R}^8$. Prove that $\mathbb{R}^8 = U \oplus W$.
		
		$\text{dim}(U + W) = \text{dim}\ U + \text{dim}\ W - \text{dim}(U \cap W)$
		
		$\text{dim}(\mathbb{R}^8) = \text{dim}\ U + \text{dim}\ W - \text{dim}(U \cap W)$
		
		$8 = 3 + 5 - \text{dim}(U \cap W)$
		
		$\text{dim}(U \cap W) = 0$
		
		$\therefore U \cap W = \{0\}$
		
		Because $U \cap W = \{0\}$, $U + W$ is a direct sum and $U \oplus W = \mathbb{R}^8$.
		
		\item[2.C.12] Suppose $U$ and $W$ are both five-dimensional subspaces of $\mathbb{R}^9$. Prove that $U \cap W \neq \{0\}$.
		
		$U + W$ is a subspace of $\mathbb{R}^9$.
		
		$\therefore \text{dim}(U + W) \leq \text{dim}(R^9)$
		
		$\text{dim}\ U + \text{dim}\ W - \text{dim}(U \cap W) = \text{dim}(U + W) \leq \text{dim}(R^9)$
		
		$5 + 5 - \text{dim}(U \cap W) \leq 9$
		
		$10 - \text{dim}(U \cap W) \leq 9$
		
		$-\text{dim}(U \cap W) \leq -1$
		
		$\text{dim}(U \cap W) \geq 1$
		
		$\therefore \text{dim}(U \cap W) \neq 0 \Rightarrow U \cap W \neq \{0\}$.
	\end{enumerate}
	
\end{document}
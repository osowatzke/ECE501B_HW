\documentclass[fleqn]{article}

\usepackage{enumitem}
\usepackage[nodisplayskipstretch]{setspace}
\usepackage{amsmath, nccmath}
\usepackage{amssymb}
\usepackage{etoolbox}
\usepackage[utf8]{inputenc}
\usepackage{tabularx}
\usepackage{array}
%\usepackage{tabu}

\newcommand{\zerodisplayskip}{
	\setlength{\abovedisplayskip}{0pt}
	\setlength{\belowdisplayskip}{0pt}
	\setlength{\abovedisplayshortskip}{0pt}
	\setlength{\belowdisplayshortskip}{0pt}
	\setlength{\mathindent}{0pt}}
	
\makeatletter
\preto\equation{\@fleqnfalse}
\makeatother

%\makeatletter
%\preto\align{\@fleqntrue}
%\makeatother

%\makeatletter
%\preto\equation{\@fleqnfalse\singlespacing}
%\patchcmd{\equation}{\doublespacing\disablespaces}
%\makeatother

\BeforeBeginEnvironment{align*}{\singlespacing}
\AfterEndEnvironment{align*}{\doublespacing\vspace{-\baselineskip}}
%\noindent\ignorespaces\disablespaces\disableparskip}

%\makeatletter
%\renewcommand{\equation{\@fleqnfalse}}{\endequation{\@fleqntrue}}
%\preto\equation{\@fleqnfalse}
%\makeatother

%\makeatletter
%\preto\align{\@fleqnflase}
%\makeatother

\title{Homework 3}
\author{Owen Sowatze}
\date{September 25, 2023}

\begin{document}
	
	\zerodisplayskip
	\doublespacing
	\maketitle
	
	\begin{enumerate}[nolistsep]
		\item[1.] Give an example of a function $\varphi : \mathbb{R}^2 \rightarrow \mathbb{R}$ such that
		
		\centerline{$\varphi(av) = a\varphi(v)$}
		
		for all $a \in \mathbb{R}$ and all $v \in \mathbb{R}^2$ but $\varphi$ is not linear.
		
		[\textit{The exercise above and the next exercise show that neither homogeneity nor additivity alone is enough to imply that a function is a linear map.}]
		
		Let $\varphi(v) = \varphi((x,y)) = \sqrt{x^2+y^2}$:
		
		$\varphi(av) = \varphi((ax,ay)) = \sqrt{(ax)^2 + (ay)^2} = \sqrt{a^2x^2 + a^2y^2}$
		
		$ = \sqrt{a^2(x^2 + y^2)} = a\sqrt{x^2+y^2} = a\varphi((x,y)) = a\varphi(v)$
		
		$\therefore \varphi$ satisfies homogeneity.
		
		Now check whether it satitisfies additivity:
		
		$\varphi(v_1+v_2) = \varphi((x_1+x_2,y_1+y_2)) = \sqrt{(x_1+x_2)^2 + (y_1+y_2)^2}$
		
		$\varphi(v_1) + \varphi(v_2) = \varphi((x_1,y_1)) + \varphi((x_2,y_2)) = \sqrt{x_1^2+y_1^2} + \sqrt{x_2^2+y_2^2}$
		
		$\varphi(v_1+v_2)$ is not equal to $\varphi(v_1) + \varphi(v_2)$ for all $v_1, v_2 \in \mathbb{R}^2$
		
		Consider $v_1 = (1, 0)$ and $v_2 = (0, 1)$:
		
		$\varphi(v_1+v_2) = \sqrt{1^2 + 1^2} = \sqrt{2}$
		
		$\varphi(v_1) + \varphi(v_2) = \sqrt{1^2 + 0^2} + \sqrt{0^2 + 1^2} = 1 + 1 = 2$
		
		$\varphi(v_1+v_2) \neq \varphi(v_1) + \varphi(v_2)$
		
		$\therefore \varphi$ does not satisfy additivity.
		
		Because $\varphi$ does not satisfy additivity, it is not linear.
		
		\pagebreak
		\item[2.] Suppose $T \in \mathcal{L}(V,W)$ is injective and $v_1,...,v_n$ is linearly independent in $V$. Prove that $Tv_1,...,Tv_n$ is linearly independent in $W$.
		
			%Consider $T(a_1v_1 + \cdots a_nv_n)$
			
			%Because $v_1,...,v_n$ is linearly independent in $V$, the only choice of \newline $a_1,...,a_n \in \mathbb{F}$ that makes $a_1v_1 + \cdots + a_nv_n = 0$ is $a_1 = \cdots = a_n = 0$
			
			Consider the choices of $a_1,...,a_n \in \mathbb{F}$ that make $a_1Tv_1 + \cdots a_nTv_n = 0$.
			
			%$a_1Tv_1 + \cdots + a_nTv_n = T(a_1v_1 + \cdots + a_nv_n)$
			
			Because $T$ is injective, the only way to make $T(a_1v_1 + \cdots + a_nv_n) = a_1Tv_1 + \cdots + a_nTv_n = 0$ is by setting $a_1v_1 + \cdots + a_nv_n = 0$.
			
			Furthermore, because $v_1,...,v_n$ are linearly independent in $V$, the only way to make $a_1v_1 + \cdots + a_nv_n = 0$ is by setting	$a_1 = \cdots = a_n = 0$.
			
			$\therefore$ the only way to make $a_1Tv_1 + \cdots + a_nTv_n = 0$ where $a_1,...,a_n \in \mathbb{F}$ is by setting $a_1 = \cdots = a_n = 0$.
			
			As a result, $Tv_1,...,Tv_n$ must be linearly independent in $W$.
	
		\item[3.] Suppose $v_1,...v_n$ spans $V$ and $T \in \mathcal{L}(V,W)$. Prove that the list\newline $Tv_1,...,Tv_n$ spans $\text{range}\ T$.

		The $\text{range}\ T$ is defined as follows:
		
		$\text{range}\ T = \{Tv : v \in V\}$
		
		Because $v_1,...,v_n$ spans $V$, every $v \in V$ can be written as a linear combination of the vectors $v_1,...,v_n$:
		
		$v = a_1v_1 + \cdots + a_nv_n$ for $a_1,...,a_n \in \mathbb{F}$
		
		Using this result, every vector in the $\text{range}\ T$ can be written as follows:
		
		$Tv = T(a_1v_1 + \cdots a_nv_n) = a_1Tv_1 + \cdots + a_nTv_n$ for $a_1,...,a_n \in \mathbb{F}$
		
		$\therefore Tv_1,...,Tv_n$ spans $\text{range}\ T$.
		
		\item[4.] Suppose $V$ is finite-dimensional and $S,T \in \mathcal{L}(V)$. Prove that $ST = I$ if any only if $TS = I$.
		
	Proof of if ($\Rightarrow$):
	
	Let $ST = I$
	
	$I$ is invertible because $IA = I$ and $AI = I$, where $A = I$.
	
	Because $I$ is invertible, $ST$ is invertible.
	
	$\therefore \exists\ U \in \mathcal{L}(V)$ such that $U(ST) = I$ and $(ST)U = I$.
	
	Let $v \in V$ and $Tv = 0$
	
	$v = Iv = U(ST)v = US(Tv) = 0$
	
	$\therefore \text{null}\ T = \{0\} \Rightarrow T$ is injective.
	
	$\text{dim}\ V = \text{dim null}\ T + \text{dim range}\ T$
	
	$\Rightarrow \text{dim range}\ T = \text{dim}\ V \Rightarrow \text{range}\ T = V \Rightarrow T$ is surjective. 
	
	Because $T$ is injective and surjective, it is invertible.
	
	%Let $w \in V$
	
	%$w = Iw = (ST)Uw = S(TUw)$
	
	%$w \in \text{range}\ S$
	
	%Because $w$ is an arbitrarily chosen vector in $V$, $\text{range}\ S = V$.
	
	%$\Rightarrow S$ is surjective.
	
	%$\text{dim}\ V = \text{dim null}\ S + \text{dim range}\ S \Rightarrow \text{dim null}\ S = 0$
	
	%$\Rightarrow \text{null}\ S = \{0\} \Rightarrow S$ is injective.
	
	%Because $S$ is injective and surjective, it is invertible.
	
	Since $T$ is invertible, we can write $ST = I$ as
	
	$(ST)T^{-1} = IT^{-1}$
	
	$S(TT^{-1}) = T^{-1}$
	
	$SI = T^{-1}$
	
	$S = T^{-1}$
	
	$TS = TT^{-1}$
	
	$TS = I$
	
	Proof of only if ($\Leftarrow$):
	
	Let $TS = I$
	
	$I$ is invertible because $IA = I$ and $AI = I$, where $A = I$.
	
	Because $I$ is invertible, $TS$ is invertible.
	
	$\therefore \exists\ U \in \mathcal{L}(V)$ such that $U(TS) = I$ and $(TS)U = I$.
	
	Let $v \in V$ and $Sv = 0$
	
	$v = Iv = U(TS)v = UT(Sv) = 0$
	
	$\therefore \text{null}\ S = \{0\} \Rightarrow S$ is injective.
	
	$\text{dim}\ V = \text{dim null}\ S + \text{dim range}\ S$
	
	$\Rightarrow \text{dim range}\ S = \text{dim}\ V \Rightarrow \text{range}\ S = V \Rightarrow S$ is surjective.
	
	Because $S$ is injective and surjective, it is invertible.
	
	Since $S$ is invertible, we can write $TS = I$ as
	
	$(TS)S^{-1} = IS^{-1}$
	
	$T(SS^{-1}) = S^{-1}$
	
	$TI = S^{-1}$
	
	$T = S^{-1}$
	
	$ST = SS^{-1}$
	
	$ST = I$
	
	\item[5.] \textbf{Linear Filters:} Filters are commonly used in various applications, such as signal processing and communications, to condition the signal(s) for a particular purpose. Linear shift-invariant (LSI) filters, a special subset of filters, are designed to accomplish various spectral filtering operations, e.g., suppressing the high-frequencies in a signal, i.e., a low-pass filter, or rejecting a particular narrow band of frequencies, i.e., a notch-filter. In the continuous time-domain, a LSI filter operation is mathematically expressed as a convolution integral,
	
		\begin{equation}
			y(t) = \int{h(t-\tau)x(\tau)d\tau}
		\end{equation}
		
		where $x(t)$ is the input signal, $h(t)$ is the filter’s impulse response function, and $y(t)$ is the filter’s output signal. In the discrete-time setting or in the sampled, digital domain, the LSI filtering operation can be expressed as a convolution sum,
	
		\begin{equation}
			y(n) = \sum_{k=0}^{N-1}{h(n-k)x(k+1)}
			\label{dt_convolution}
		\end{equation}
		
		%\setlength{\abovedisplayskip}{0pt}
		%\setlength{\belowdisplayskip}{0pt}
		%\setlength{\abovedisplayshortskip}{0pt}
		%\setlength{\belowdisplayshortskip}{0pt}
		%\setlength{\mathindent}{0pt}
	
		where $x(n)$ is an input signal sequence of length N that can be expressed as an N-dimensional vector $x = (x_1, x_2, ..., x_N)^T$, the filter impulse response $h(n)$ is of length $M$, expressed as an M-dimensional vector $h = (h_1, h_2, ..., h_M)$ (typically, $M < N$), and the output signal signal y of length N, $y = (y_1, y_2, ..., y_N)^T$.
		
		\pagebreak
		\begin{enumerate}
			\item[a)] Show that a LSI filter, as defined in Equation (\ref{dt_convolution}), is a linear map $T : V \rightarrow W$, where $V = W = \mathbb{R}^N$ and $x,y \in \mathbb{R}^N$.
			
			For the LSI filter to be a linear map it must satisfy additivity and homogeneity.
			
			Let:
			
			\begin{align*}
				T(x) = y(n) = \sum_{k=0}^{N-1}{h(n-k)x(k+1)}
			\end{align*}
			
			\begin{align*}
				T(\bar{x}) = \bar{y}(n) = \sum_{k=0}^{N-1}{h(n-k)\bar{x}(k+1)}
			\end{align*}
			
			\begin{align*}
				T(x + \bar{x}) = \sum_{k=0}^{N-1}{h(n-k)\{x(k+1) + \bar{x}(k+1)\}}
			\end{align*}
			
			\begin{align*}
				 = \sum_{k=0}^{N-1}{h(n-k)x(k+1)} + \sum_{k=0}^{N-1}{h(n-k)\bar{x}(k+1)} = T(x) + T(\bar{x})
			\end{align*}
			
			$\Rightarrow$ The LSI filter satisfies additivity.
			
			Let $\lambda \in \mathbb{R}$ and
			
			\begin{align*}
				T(x) = y(n) = \sum_{k=0}^{N-1}{h(n-k)x(k+1)}
			\end{align*}
			
			\begin{align*}
				T(\lambda x) = \sum_{k=0}^{N-1}{h(n-k)\{\lambda x(k+1)\}}
			\end{align*}
			
			\begin{align*}
			= \lambda\sum_{k=0}^{N-1}{h(n-k) x(k+1)} = \lambda T(x)
			\end{align*}
			
			$\Rightarrow$ The LSI filter satisfies homogeneity.
			
			Because the LSI filter satisfies additivity and homogeneity, it is a linear map.
			
			\pagebreak
			\item[b)] Express the LSI filter's linear map $H$ as an $N \times N$ matrix $\mathbf{H}$ such that
			
			\begin{center}
				$y = \mathbf{H}x$
			\end{center}
			
			Here you can assume that $h(m) = 0$ for $m \not\in \{1,2,3,...,M\}$.
			
			Hint: Expand an output vector that results from applying the map (operating) on a basis vector of $V$, in the basis of $V$. The coefficients of this expansion produce the column vectors of the matrix. You can use the standard basis.

			Let the $v_1,...,v_n$ be the standard basis vectors of $V$, i.e.
			
			$v_1 = (1, 0,...,0)^T$
			
			$v_2 = (0, 1,...,0)^T$
			
			$\vdots$
			
			$v_N = (0, 0,...,1)^T$
			
			Now solve for the columns of $\mathbf{H}$ by finding $T(v_1),...,T(v_2)$
			
			\begin{align*}
				T(v_1) = \sum_{k=0}^{N-1}{h(n-k)\delta(k)} = h(n) = (h_1,...,h_M,0,...,0)^T
			\end{align*}
			
			\begin{align*}
				T(v_2) = \sum_{k=0}^{N-1}{h(n-k)\delta(k-1)} = h(n-1) = (0, h_1,...,h_M,0,...,0)^T
			\end{align*}
			
			$\vdots$
			
			\begin{align*}
				T(v_N) = \sum_{k=0}^{N-1}{h(n-k)\delta(k-N)} = h(n-N) = (0, 0,...,0,h_1)^T
			\end{align*}
			
			\newpage
			$\therefore \mathbf{H}$ can be expressed as:
			
			\begin{align*}
				\mathbf{H} =
				\begin{bmatrix}
					h_1 		& 0 			& \cdots 	& 0 			& 0\\
					h_2 		& h_1 		& \cdots 	& 0 			& 0\\
					\vdots 	& \vdots 	& \ddots 	& \vdots 	& \vdots\\
					h_M 		& h_{M-1} 	& \cdots 	& 0 			& 0\\
					0 		& h_M		& \cdots 	& 0 			& 0\\
					\vdots 	& \vdots 	& \ddots 	& \vdots 	& \vdots\\
					0		& 0			& \cdots 	& h_1		& 0\\       
					0 		& 0			& \cdots 	& h_2		& h_1\\
				\end{bmatrix}
			\end{align*}
			
			\begin{align*}
				T(v_N) = \sum_{k=0}^{N-1}{h(n-k)\delta(k-N)} = h(n-N) = (0, 0,...,0,h_1)^T
			\end{align*}
			
			\begin{align*}
				T(v_N) = \sum_{k=0}^{N-1}{h(n-k)\delta(k-N)} = h(n-N) = (0, 0,...,0,h_1)^T
			\end{align*}
		\end{enumerate}
		\item[6.] \textbf{Optical Resonators}: A simple model of light propagation treats it as a collection of geometric rays. There are several mathematical formalisms for dealing with this model, but one is particularly relevant to the material in ECE 501B.

			We begin by assuming that the system is axially symmetric. That is, the light propagates generally along the z-axis, and there is no appreciable difference between the x- and y- directions. Thus, we can take the full 3-dimensional (in the geometric sense) propagation and treat it as a 2-dimensional propagation.

			A ray passing through any plane perpendicular to the z-axis can be characterized by an ordered pair of numbers, usually designated $(r, r')$. The first number, $r$, represents the transverse location of the ray (its distance from the z-axis). The second number, $r'$, represents the slope of the ray with respect to the z-axis.
			
			For various optical components (thin lenses, mirrors, propagation through free space, etc.), it is possible to define linear maps between an input ray and the output ray after interaction with the component. Propagating the ray through the entire system is then just a matter of sequentially applying the appropriate maps. (Since the output of any map is another ray, i.e., the output vector space is the same as the input vector space, then the sequential application of an arbitrary number of maps is well-defined.)
			
			These maps take the form of $(2 \times 2)$ matrices. Below are the matrix representations of two maps representing different components.
			
			\begingroup
			\renewcommand*{\arraystretch}{0.6}
			\begin{itemize}
				\item
					\begin{tabularx}{0.85\textwidth}{>{\hsize=0.75\hsize}X >{\hsize=0.25\hsize}X}
						Propagation a distance d down the z-axis: & $\begin{bmatrix}
							1 & d\\
							0 & 1
							\end{bmatrix}$
					\end{tabularx}
					
				\item
					\begin{tabularx}{0.85\textwidth}{>{\hsize=0.75\hsize}X  >{\hsize=0.25\hsize}X}
						Reflection off of a concave spherical mirror: & $\begin{bmatrix}
							 1   & 0\\
							-2/R & 1
							\end{bmatrix}$
					\end{tabularx}
					
					\begin{tabularx}{0.85\textwidth}{>{\hsize=0.75\hsize}X  >{\hsize=0.25\hsize}X}
						$(R = \text{radius of curvature})$ &\\						
					\end{tabularx}
			\end{itemize}
			\endgroup
				
			Now, an optical resonator is an optical system that is designed to repeatedly cycle rays back and forth through the system. Optical resonators are used in a variety of applications, but are most famous as one of the critical components in a laser system
			
			\item[a)] What is the domain of $r$? What is the domain of $r'$? The set of all rays corresponds to what vector space? What is the dimension of this vector space?
			
			$r \in \mathbb{R}$
			
			$r' \in \mathbb{R}$
			
			$V \in \mathbb{R}^2$
			
			$dim(V) = 2$
			
			\item[b)] Consider an optical resonator consisting of two concave spherical mirrors (with radius of curvature $R$) facing each other and separated by a distance of $D$. Using the relevant matrices, determine the matrix representation of the linear map that maps an input ray to an output ray after one complete round trip through the resonator.
			
			\pagebreak
			Let $T$ be the linear map that maps an input ray to an output ray after propagation a distance $D$ down the z-axis.
			
			\begin{align*}
				\mathcal{M}(T) = \begin{bmatrix}1 & D\\ 0 & 1\end{bmatrix}
			\end{align*}
			
			Let $S$ be the linear maps that maps an input ray to an output ray after reflection off a concave spherical mirror (with radius of curvature $R$).
			
			\begin{align*}
				\mathcal{M}(S) = \begin{bmatrix} 1 & 0\\ -\frac{2}{R} & 1 \end{bmatrix}
			\end{align*}
			
			Let $R_1$ be the linear map that maps an input ray to an output ray after one round trip through the resonator.
			
			$\mathcal{M}(R_1) = \mathcal{M}(S(T(ST))) = \mathcal{M}(S)(\mathcal{M}(T)(\mathcal{M}(S)\mathcal{M}(T)))$
			
			$ = (\mathcal{M}(S)\mathcal{M}(T))(\mathcal{M}(S)\mathcal{M}(T))$
			
			Start by finding $\mathcal{M}(S)\mathcal{M}(T)$:
			
			\begin{align*}
				\mathcal{M}(S)\mathcal{M}(T) = \begin{bmatrix} 1 & 0\\ -\frac{2}{R} & 1 \end{bmatrix} \begin{bmatrix}1 & D\\ 0 & 1\end{bmatrix} = \begin{bmatrix}1 & D\\ -\frac{2}{R} & -\frac{2D}{R} + 1\end{bmatrix}
			\end{align*}
			
			Now solve for $\mathcal{M}(R_1)$:
			
			\begin{align*}
				\mathcal{M}(R_1) = \begin{bmatrix}1 & D\\ -\frac{2}{R} & -\frac{2D}{R} + 1\end{bmatrix}\begin{bmatrix}1 & D\\ -\frac{2}{R} & -\frac{2D}{R} + 1\end{bmatrix}
			\end{align*}
			
			\begin{align*}
				= \begin{bmatrix}1 - \frac{2D}{R} & D - \frac{2D^2}{R} + D\\ -\frac{2}{R} - \frac{2}{R}(-\frac{2D}{R} + 1) & -\frac{2D}{R} + (-\frac{2D}{R}+1)(-\frac{2D}{R}+1)\end{bmatrix}
			\end{align*}
			
			\begin{align*}
				= \begin{bmatrix}1 - \frac{2D}{R} & 2D - \frac{2D^2}{R}\\ -\frac{4}{R} +\frac{4D}{R^2} & -\frac{2D}{R} + \frac{4D^2}{R^2} -\frac{4D}{R} + 1\end{bmatrix} = \begin{bmatrix}1 - \frac{2D}{R} & 2D - \frac{2D^2}{R}\\ -\frac{4}{R} + \frac{4D}{R^2} & 1 - \frac{6D}{R} + \frac{4D^2}{R^2}\end{bmatrix}
			\end{align*}
			
			\item[c)] A stable resonator is one where, upon propagation, any input ray is eventually mapped back onto itself. Use your result from part (b) to show that, for a resonator of this type, there is no system configuration (other than the physically uninteresting $D = 0$, $R = \pm \infty$) that maps all rays onto themselves after a single round trip.
			
			\pagebreak
			For the input to the system to be the same as the output, the matrix representation of $R_1$ must be given as follows:
			
			\begin{align*}
				\mathcal{M}(R_1) = \begin{bmatrix}1 & 0\\ 0 & 1\end{bmatrix}
			\end{align*}
			
			To make the terms on the main diagonal $1$, $D = 0$ or $R = \pm\infty$. However, only $R = \pm\infty$ will make the term lower left corner $0$. Furthermore, only $D = 0$ will make the term in the upper right corner $0$. $\therefore$ the system is only stable if $D=0$ and $R=\pm\infty$.			
			
			\item[d)] Now find the map for two round trips (don’t repeat work – be clever). Using this map, show that the resonator with $D = R$ is stable.
			
			Let $R_2$ be the system that map an input ray to an output ray after two complete round trips through the resonator.
			
			$\mathcal{M}(R_2) = \mathcal{M}(R_1)\mathcal{M}(R_1)$
			
			Express $\mathcal{M}(R_1)$ as follows:
			
			\begin{align*}
				\mathcal{M}(R_1) = \begin{bmatrix}A_{11} & A_{12}\\ A_{21} & A_{22}\end{bmatrix}
			\end{align*}
			
			\begin{align*}
				\mathcal{M}(R_2) = \mathcal{M}(R_1)\mathcal{M}(R_1) = \begin{bmatrix}A_{11} & A_{12}\\ A_{21} & A_{22}\end{bmatrix}\begin{bmatrix}A_{11} & A_{12}\\ A_{21} & A_{22}\end{bmatrix}
			\end{align*}
			
			\begin{align*}
				= \begin{bmatrix}A_{11}^2 + A_{12}A_{21} & A_{11}A_{12} + A_{12}A_{22}\\ A_{21}A_{11} + A_{22}A_{21} & A_{21}A_{12} + A_{22}^2\end{bmatrix}
			\end{align*}
			
			where:
			
			$A_{11} = 1 - \frac{2D}{R}$
			
			$A_{12} = 2D - \frac{2D^2}{R}$
			
			$A_{21} = -\frac{4}{R} + \frac{4D}{R^2}$
			
			$A_{22} = 1 - \frac{6D}{R} + \frac{4D^2}{R^2}$
			
			\pagebreak			
			If $D = R$
			
			$A_{11} = 1 - \frac{2R}{R} = 1 - 2 = -1$
			
			$A_{12} = 2R - \frac{2R^2}{R} = 2R - 2R = 0$
			
			$A_{21} = -\frac{4}{R} + \frac{4R}{R^2} = -\frac{4}{R} + \frac{4}{R} = 0$
			
			$A_{22} = 1 - \frac{6R}{R} + \frac{4R^2}{R^2} = 1 - 6 + 4 = -1$
			
			Substituting $A_{11}, A_{12}, A_{21}, A_{22}$ into the matrix representation of $R_2$
			
			\begin{align*}
				\mathcal{M}(R_2) = \begin{bmatrix}(-1)^2 + 0(0) & (-1)0 + 0(-1)\\ 0(-1) + (-1)0 & 0(-1) + (-1)^2\end{bmatrix} = \begin{bmatrix}1 & 0\\ 0 & 1\end{bmatrix}
			\end{align*}
			
			Because the matrix representation of $R_2$ is the identity matrix, the optical resonator with $D=R$ is stable.
			
	\end{enumerate}
\end{document}
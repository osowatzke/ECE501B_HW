\documentclass{article}

\usepackage{geometry}
\usepackage{amsmath}
\usepackage{amssymb}
\usepackage{graphicx}
\usepackage{enumitem}
\usepackage[nodisplayskipstretch]{setspace}

\title{Homework 1}
\author{Owen Sowatzke}
\date{September 6, 2023}

\begin{document}

	\setlength{\abovedisplayskip}{0pt}
	\setlength{\belowdisplayskip}{0pt}
		\setlength{\abovedisplayshortskip}{0pt}
	\setlength{\belowdisplayshortskip}{0pt}
	\doublespacing
	\maketitle
	
	\begin{enumerate}[nolistsep]
	
		\item[1.] For each of the following subsets of $\mathbb{F}^3$, determine whether it is a subspace of $\mathbb{F}^3$:
		
		\begin{enumerate}[nolistsep]
		
			\item ${\{(x_1, x_2, x_3) \in \mathbb{F}^3 : x_1 + 2x_2 + 3x_3 = 0 \}}$
			
			Let $U = {\{(x_1, x_2, x_3) \in \mathbb{F}^3 : x_1 + 2x_2 + 3x_3 = 0 \}}$
			
			Check whether $U$ contains the additive identity:
			
			Consider $(x_1, x_2, x_3) = (0, 0, 0)$
			
			$x_1 + 2x_2 + 3x_3 = 0 + 2 \cdot 0 + 3 \cdot 0 = 0$
			
			$\therefore 0 \in U$
			
			Check whether $U$ is closed under addition:
			
			Let $u,v \in U$
			
			Consider $u + v = (u_1, u_2, u_3) + (v_1, v_2, v_3) = (u_1 + v_1, u_2 + v_2, u_3 + v_3)$
			
			$(u_1 + v_1) + 2(u_2 + v_2) + 3(u_3 + v_3)$
			
			$ = u_1 + v_1 + 2u_2 + 2v_2 + 3u_3 + 3v_3$
			
			$ = (u_1 + 2u_2 + 3u_3) + (v_1 + 2v_2 + 3v_3)$
			
			$ = 0 + 0 = 0$
			
			$\therefore u + v \in U$
			
			Closed whether $U$ is closed under scalar multiplication:
			
			Let $u \in U$ and $a \in \mathbb{F}$
			
			Consider $au = (au_1, au_2, au_3)$
			
			$(au_1) + 2(au_2) + 3(au_3)$
			
			$ = a(u_1 + 2u_2 + 3u_3)$
			
			$ = a \cdot 0 = 0$
			
			$\therefore au \in U$
			
			Because $U$ contains the additive identity, is closed under addition, and is closed under scalar multiplication, $U$ is a subspace of $\mathbb{F}^3$
			
			\item ${\{(x_1, x_2, x_3) \in \mathbb{F}^3 : x_1 + 2x_2 + 3x_3 = 4 \}}$
			
			Let $U = {\{(x_1, x_2, x_3) \in \mathbb{F}^3 : x_1 + 2x_2 + 3x_3 = 4 \}}$
			
			Check whether $U$ contains the additive identity:
			
			Consider $(x_1, x_2, x_3) = (0, 0, 0)$
			
			$x_1 + 2x_2 + 3x_3 = 0 + 2 \cdot 0 + 3 \cdot 0 = 0 \neq 4$
			
			$\therefore 0 \notin U$
			
			Because $U$ does not contain the additive identity, it is not a subspace of $\mathbb{F}^3$

			\item ${\{(x_1, x_2, x_3) \in \mathbb{F}^3 : x_1x_2x_3 = 0 \}}$
			
			Let $U = {\{(x_1, x_2, x_3) \in \mathbb{F}^3 : x_1x_2x_3 = 0 \}}$
			
			Check whether $U$ is closed under addition:
			
			Consider $u = (1, 1, 0) \in U$ and $v = (0, 0, 1) \in U$.
			
			$u + v = (1, 1, 1)$
			
			$1 \cdot 1 \cdot 1 = 1 \neq 0$
			
			$\therefore u + v \notin U$
			
			Because $U$ is not closed under addition, it is not a subspace of $\mathbb{F}^3$
			
			\item ${\{(x_1, x_2, x_3) \in \mathbb{F}^3 : x_1 = 5x_3 \}}$
			
			Let $U = \{(x_1, x_2, x_3) \in \mathbb{F}^3 : x_1 = 5x_3 \}$
			
			Check whether $U$ contains the additive identity:
			
			Consider $(x_1, x_2, x_3) = (0, 0, 0)$
			
			$x_1 = 5x_3$
			
			$0 = 5 \cdot 0$
			
			$\therefore 0 \in U$
			
			Check whether $U$ is closed under addition:
			
			Let $u, v \in U$
			
			Consider $u + v = (u_1, u_2, u_3) + (v_1, v_2, v_3) = (u_1 + v_1, u_2 + v_2, u_3 + v_3)$
			
			$u_1 + v_1 = 5u_3 + 5v_3 = 5(u_3 + v_3)$
			
			$\therefore u + v \in U$
			
			Check whether $U$ is closed under scalar multiplication:
			
			Let $u \in U$ and $a \in \mathbb{F}$
			
			$au = (au_1, au_2, au_3)$
			
			$au_1 = a(5u_3) = 5(au_3)$
			
			$\therefore au \in U$
			
			Because $U$ contains the additive identity, is closed under addition, and is closed under scalar multiplication, $U$ is a subspace of $\mathbb{F}^3$
			
		\end{enumerate}
		
		\item[7.] Give an example of a nonempty subset $U$ of $\mathbb{R}^2$ such that U is closed under addition and under taking additive inverses (meaning $-u \in U$ whenever $u \in U$), but $U$ is not a subspace of $\mathbb{R}^2$.
		
		Let $U = \{(x_1, x_2) \in \mathbb{R}^2 : x_1, x_2 \in \mathbb{Z}\}$
		
		Let $u, v \in U$.
		
		$u + v = (u_1, u_2) + (v_1, v_2) = (u_1 + v_1, u_2 + v_2)$
		
		For $u_1, v_1 \in \mathbb{Z}$, $u_1 + v_1 \in \mathbb{Z}$.
		
		For $u_2, v_2 \in \mathbb{Z}$, $u_2 + v_2 \in \mathbb{Z}$
		
		$\therefore u + v \in U$ (i.e. $U$ is closed under addition)
		
		$-u = (-u_1, -u_2)$
		
		For $u_1 \in \mathbb{Z}$, $-u_1 \in \mathbb{Z}$
		
		For $u_2 \in \mathbb{Z}$, $-u_2 \in \mathbb{Z}$
		
		$\therefore -u \in U$ (i.e. $U$ is closed under taking additive inverses)
		
		Check whether $U$ is closed under scalar multiplication:
		
		Consider $a = \frac{1}{2} \in \mathbb{R}$ and $u = (1, 1) \in \mathbb{R}^2$
		
		$au = \frac{1}{2}(1, 1) = (\frac{1}{2}, \frac{1}{2}) \not\in U$
		
		$\therefore U$ is not closed under scalar multiplication, and $U$ is not a subspace of $\mathbb{R}^2$
		
		\item[8.] Give an example of a nonempty subset of $U$ such that U is closed under scalar multiplication, but $U$ is not a subspace of $\mathbb{R}^2$.
		
		Let $U = \{(x_1, x_2) \in \mathbb{R}^2 : x_1^2 = x_2^2\}$
		
		Let $u \in U$ and $a \in \mathbb{R}^2$
		
		$(ax_1)^2 = a^2x_1^2 = a^2x_2^2 = (ax_2)^2 \Rightarrow au \in U$
		
		Let $u = (1, -1) \in U$ and $v = (1, 1) \in U$.
		
		$(1 + 1)^2 = 2^2 = 4 \neq 0 = 0^2 = (1 - 1)^2$
		
		$\therefore U$ is not closed under addition, and $U \not\subset \mathbb{R}^2$
		
		\item[19.] Prove or give a counterexample: if $U_1, U_2, W$ are subspaces of $V$ such that
		
\centerline{$U_1 + W = U_2 + W$,}

then $U_1 = U_2$. 
		
		Consider the following counterexample:

		Let $U_1 = \{(x, 0) \in \mathbb{F}^2 : x \in \mathbb{F}\}$ and $U_2 = \{(0, y) \in \mathbb{F}^2 : y \in \mathbb{F}\}$
		
		Clearly, $U_1 \neq U_2$.
		
		Let $W = \{(x, y) \in \mathbb{F}^2 : x, y \in \mathbb{F}\}$
		
		Then $U_1 + W = U_2 + W = \{(x, y) \in \mathbb{F}^2 : x, y \in \mathbb{F}\}$
		
		\item[23.] Prove or give a counterexample: if $U_1, U_2, W$ are subspaces of $V$ such that
		
		\centerline{$V = U_1 \oplus W$ and $V = U_2 \oplus W$}
		
		then $U_1 = U_2$.
		
	\end{enumerate}
\end{document}